\chapter{Basics 1}\label{basics-1}

\section{Welcome to our course!}\label{welcome-to-our-course}

Now you are ready to proceed to the main part of the tree!

We are happy that you have chosen our Russian course. Just to make it
clear, we are using American English in this course---but don't worry,
we will accept all versions of English where appropriate. Just be
careful around expressions like ``bathroom'' or ``1st floor'', because
these may mean different things than what you are used to.

As for Russian, we teach the standard language, which is based on the
variation spoken around Moscow and Saint Petersburg, and we stick to the
usage typical of these cities. Do not worry, though: for more than one
reason Russian is rather uniform over the territory of Russian (still,
there is some variation in pronunciation and a few items of everyday
vocabulary). We try to stay neutral in style, with occasional trips into
formal and informal language.

\section{Cases and word order}\label{cases-and-word-order}

Russian is an inflected language, so the forms of nouns and modifying
adjectives correspond to their role in the sentence.

These forms are called \textbf{cases}. Russian has 6 cases: Nominative,
Accusative, Genitive, Prepositional, Dative and Instrumental. The
Nominative is the dictionary form; as for the others, we are going to
cover them gradually, one by one.

This allows for a more loose word order. But not random! A typical word
order is \emph{subject---verb---object}. ``Old'' information (the things
you tell \emph{about}) is normally closer to the beginning of the
sentence which is probably why pronouns are often found closer to the
beginning of a sentence than a noun would be :

\begin{itemize}
\tightlist
\item
  \emph{I know him.} $ \rightarrow$ Я \textbf{ег\'{о}} зн\'{а}ю.
\item
  \emph{I know Maria.} $ \rightarrow$ Я зн\'{а}ю \textbf{Мар\'{и}ю}.
\end{itemize}

That includes words like ``here'', ``in this way'', ``then'' and so on.

Unlike English, adverbs are NOT universally grouped at the end. So pay
attention to the typical positions for the expressions of time, place
and manner. Eg. ``very much'' is typically in the end-position in
English, but in Russian it is just before the thing that is ``very'' or
``very much'':

\begin{itemize}
\tightlist
\item
  \emph{She likes to read \textbf{very much}} = Он\'{а} \textbf{\'{о}чень} л\'{ю}бит
  чит\'{а}ть
\end{itemize}

Good luck!

\section{Vowel reduction}\label{vowel-reduction}

Like in English, vowel letters aren't \emph{all} pronounced just like in
the alphabet. In Russian, unstressed syllables have vowels
\emph{reduced}:

\begin{itemize}
\tightlist
\item
  \textbf{А and О} become the same uh-sound
\item
  \textbf{И and Е} (Э) become the same sound similar to ``i'' in ``hit''
\item
  \textbf{Я} actually becomes an i-like sound, not an uh-like (except in
  a few words). This also affects ``а'' after ч,ш,щ,ж or ц in many words
  (sadly, not all).
\end{itemize}

So, when a vowel is not stressed, it becomes weaker, somewhat shorter,
and also some vowels become indistinguishable.

The unstressed syllable is strongest just before the stress. In all
other places it is even weaker than that (though, some long words,
especially compounds, may acquire a secondary stress). This makes the
system different from the English one, where stronger and weaker
syllables tend to alternate.

\section{More on the case system}\label{more-on-the-case-system}

For now, we only study simple sentences that either use the dictionary
form, the \emph{Nominative} case, or use the \emph{Accusative} (direct
object of an action), which has the same form for many classes of nouns.

The case is defined by its use. Nevertheless, these forms have names,
usually calques from Latin that reflect some typical use (but not the
only one):

\begin{itemize}
\tightlist
\item
  Nominative (subject)
\item
  Accusative (direct object)
\item
  Genitive (``of'' something)
\item
  Prepositional (place or topic)
\item
  Dative (recipient, ``indirect'' object)
\item
  Instrumental (means of action)
\end{itemize}

As you can see, these names are of little use until you know what
sentence, verb or preposition requires that you use that particular
form.

\begin{itemize}
\tightlist
\item
  some nouns of foreign origin are \emph{indeclinable}, i.e. all their
  forms are the same. This includes words like метро, Дженни or кафе.
\end{itemize}
