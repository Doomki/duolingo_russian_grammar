\chapter{Partitive}\label{partitive}

As you know, the Genitive case has \emph{lots} of uses in Russian.

One of them expresses \textbf{an amount} of something:

\begin{itemize}
\tightlist
\item
  чашка чая = a cup of tea
\item
  тарелка риса = a plate of rice
\item
  корзина яблок = a basket of apples
\end{itemize}

With mass nouns it is also used to express ``some'' unspecified amount
of that stuff when used instead of the Accusative:

\begin{itemize}
\tightlist
\item
  Я хочу воды = I want (some) water.
\item
  Дайте, пожалуйста, риса. = Could I have some rice, please? (literally,
  ``Give me, please, some rice'').
\item
  Хочешь сока? = Want some juice?
\end{itemize}

Note that this usage is only characteristic for situations when you ask
or hypothesize about using ``some or other amount'' of a substance. You
cannot actually say that you are drinking ``воды'' right now---but you
can say that you want some (or that you sipped some in the past---with a
perfective$^1$, of course).

\section{чашка
чаю}\label{ux447ux430ux448ux43aux430-ux447ux430ux44e}

«Чай» has an alternative Partitive form «чаю»:

\begin{itemize}
\tightlist
\item
  Хочешь чашечку чаю? = Want cup of tea?
\end{itemize}

It is optional. Actually, many short masculine nouns that denote
substances used to have such form. However, «чай» is, probably, the only
one where the form is immensely popular in spoken speech and does not
sound old-fashioned or downright archaic.

\section{стакан}\label{ux441ux442ux430ux43aux430ux43d}

Russian differentiates between a number of drinking vessels.
\textbf{Стак\'{а}н} is what you call a ``glass'' in English: typically, a
cylindrical vessel made of glass, with no handle. However, if you mean a
measurement unit (quite popular in cooking), it corresponds to the
English word ``cup''. In Russian you use not a cup or rice or flour but
a ``glass'' of rice or flower.

\begin{itemize}
\tightlist
\item
  a beer or a wine glass is «бок\'{а}л»
\item
  a smaller wine glass is «р\'{ю}мка»
\end{itemize}

\begin{center}\rule{0.5\linewidth}{\linethickness}\end{center}

$^1$ Perfective is an \emph{aspect}. Russian has verbs of two flavors:
those that denote ``processes'' and those that mean ``events'' (events
are never used in the present). I would argue that aspect is the main
culprit for consumption verbs here. You can want ``воды'' forever, but
you aren't \emph{``drinking''} it at any specific moment. Semantically,
``some'' water only becomes a real amount when you are done, not while
you are still at it.
