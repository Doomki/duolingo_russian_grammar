\chapter{The Accusative: the Direct
Object}\label{the-accusative-the-direct-object}

\section{Accusative}\label{accusative}

Until now, you've been using the base form of the word in sentence like
«Он ест \textbf{яблоко}».

Actually, whenever a verb, like ``read'', ``cut'' or ``want'' acts
\emph{directly} on some noun, the latter is a \emph{direct object}. Such
nouns take the Accusative case.

\section{Formation}\label{formation}

Only feminine nouns ending in \textbf{-а} / \textbf{-я} have a separate
form. «Мама» is a good example of this class :

\begin{itemize}
\tightlist
\item
  м\'{а}ма $ \rightarrow$ м\'{а}м\textbf{у}
\end{itemize}

Neuter nouns and feminine nouns with a final \textbf{-ь} (e.g., «м\'{ы}шь»)
use the Nominative form.

Now we are left with masculine nouns ending in a consonant (сок,
медв\'{е}дь, брат). They use the same form as in Nominative or Genitive:

\begin{itemize}
\tightlist
\item
  living beings (``animate'') copy the Genitive
\item
  objects (``inanimate'') stay Nominative
\item
  in plural this rule applies to all types of nouns
\end{itemize}

\begin{longtable}[]{@{}llll@{}}
\toprule
\textbf{-а/-я} & \textbf{---} (masc.) & \textbf{neuter} & \textbf{-ь}
(fem.)\tabularnewline
\midrule
\endhead
\textbf{-у/-ю} & Nom. / Gen. & Nominative & Nominative\tabularnewline
\bottomrule
\end{longtable}

\emph{With ``substances''(mass nouns)} \textbf{\emph{Genitive}}
\emph{may be used instead to convey a meaning of ``some'' quantity.}

Verbs that take a direct object are called \emph{transitive}.
Unfortunately, some verbs that are transitive in Russian are not
transitive in English (``wait'') and vice versa (``like'').

\section{I want some}\label{i-want-some}

Russian has two main verb form patterns, which we are going to introduce
soon. Unfortunately, the verb «хот\'{е}ть»(to want) is irregular and mixes
both. On a brighter note, it is a very common verb, so you'll memorize
it eventually.

The other notable thing is that it does not have a strong connotation of
`need', unlike the English verb '``want''. Similarly, the Russian verb
for ``give''(д\'{а}ть) is totally OK for polite requests. Just use it with
«пож\'{а}луйста».

\begin{itemize}
\tightlist
\item
  \emph{the one the `giving' is directed towards is NOT a direct object
  in Russian. It is called an indirect object and takes the Dative.
  We'll deal with it later.}
\end{itemize}
