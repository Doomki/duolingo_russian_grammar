\chapter{Verbs in the Present 1}\label{verbs-in-the-present-1}

\section{Е- and И-
conjugation}\label{ux435--and-ux438--conjugation}

The verbs in Russian change according to person and number. Each form
has a different ending. There are only two patterns (apart from some
phonetic changes).

\begin{longtable}[]{@{}lll@{}}
\toprule
\begin{minipage}[b]{0.32\columnwidth}\raggedright\strut
\strut
\end{minipage} & \begin{minipage}[b]{0.32\columnwidth}\raggedright\strut
endings\strut
\end{minipage} & \begin{minipage}[b]{0.32\columnwidth}\raggedright\strut
Е- / И- examples\strut
\end{minipage}\tabularnewline
\midrule
\endhead
\textbf{я} & -\textbf{ю} (у) & чит\'{а}ю, пиш\'{у} / говор\'{ю}, в\'{и}жу\tabularnewline
\textbf{ты} & -\textbf{ешь} / -\textbf{ишь} & чит\'{а}ешь, п\'{и}шешь /
говор\'{и}шь, в\'{и}дишь\tabularnewline
\textbf{он/он\'{а}} & -\textbf{ет} / -\textbf{ит} & чит\'{а}ет, п\'{и}шет / говор\'{и}т,
в\'{и}дит\tabularnewline
\begin{minipage}[t]{0.32\columnwidth}\raggedright\strut
\strut
\end{minipage} & \begin{minipage}[t]{0.32\columnwidth}\raggedright\strut
\strut
\end{minipage} & \begin{minipage}[t]{0.32\columnwidth}\raggedright\strut
\strut
\end{minipage}\tabularnewline
\textbf{мы} & -\textbf{ем} / -\textbf{им} & чит\'{а}ем, п\'{и}шем / говор\'{и}м,
в\'{и}дим\tabularnewline
\textbf{вы} & -\textbf{ете} / -\textbf{ите} & чит\'{а}ете, п\'{и}шете /
говор\'{и}те, в\'{и}дите\tabularnewline
\textbf{он\'{и}} & -\textbf{ют}(ут) / -\textbf{ят} (ат) & чит\'{а}ют, п\'{и}шут /
говор\'{я}т, в\'{и}дят\tabularnewline
\bottomrule
\end{longtable}

We will learn these one by one. There are only \textbf{four} stems with
irregular conjugation. The verbs \emph{хот\'{е}ть, дать, есть, беж\'{а}ть} and
all their derivatives do not strictly follow any of the 2.

Note that if the endings are stressed, Ё replaces Е. Fortunately, a
non-past form has only 2 options:

\begin{itemize}
\tightlist
\item
  fixed stress -- on the stem (чит\'{а}ю, чит\'{а}ете, в\'{и}жу, в\'{и}дит) or on the
  ending (сто\'{ю}, сто\'{и}т, сто\'{и}шь)
\item
  ``я''-form is has a stressed ending (Я пиш\'{у}). The stress is on the
  stem in the other forms (ты п\'{и}шешь, она п\'{и}шет..)
\end{itemize}

\emph{A verb uses one stem to form Infinitive and Past tense forms. It
uses the 2nd one, similar, for non-past and imperative. Thus, as a rule
you cannot predict all forms from the infinitive. You can make a guess,
though.}

\section{Meals}\label{meals}

In this course we use the American English definitions:

\begin{itemize}
\tightlist
\item
  \textbf{з\'{а}втрак} = \emph{breakfast}, a morning meal
\item
  \textbf{об\'{е}д} = \emph{lunch}, a midday meal
\item
  \textbf{\'{у}жин} = \emph{dinner}, an evening meal
\end{itemize}
