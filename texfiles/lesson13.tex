\chapter{The Infinitive, Likes and
Dislikes}\label{the-infinitive-likes-and-dislikes}

\section{I like/I love ?}\label{i-likei-love}

In Russian, you can express liking things and activities pretty much the
same way as in English, with similar verbs. The usage differs a bit,
though.

A a rule of thumb, \textbf{«Я любл\'{ю}»} means \emph{``I love''} only when
directed at a single person (or animal). Otherwise, it's just ``I
like''.

\begin{itemize}
\item
  ``LOVE'' \textbf{люб\'{и}ть} means a stable, lasting feeling (note the
  phonetic change for the 1st person singular: ``люблю''). A normal,
  transitive verb, i.e. used with the Accusative. Use it for
  \textbf{\emph{loving}} an individual or \emph{liking some
  things/people/activity in general} (verbs take infinitive). Very much
  preferred in negations of such activities (i.e. ``don't like to
  wait'')
\item
  ``LIKE'' \textbf{нр\'{а}виться} means moderate ``liking'' something or
  someone, often something specific. Not transitive! The thing liked is
  the subject, acting indirectly on a person: «Мне нр\'{а}вится стол» =
  \emph{I like the table.}
\item
  note that «Мне нр\'{а}вится стол» works in a similar way to the English
  verb ``to seem'': \emph{``The table seems good to me''}. The sentence
  is built as though the table ``transmits'' the feeling towards you.
  While rare in English, in Russian, this is pretty typical for feelings
  and experience to be expressed that way («Мне хорош\'{о}»).
\end{itemize}

Infinitive «нр\'{а}виться» and 3rd person singular «нр\'{а}вится» are pronounced
exactly the same, however, for the sake of consistency they are spelt
differently (most infinitives end in «-ть», so -ть + ся = -ться,
naturally)

When you refer to generic things and activities, both verbs can be used
but «люб\'{и}ть» is mildly more useful.

\section{May I?}\label{may-i}

Possibility and/or permission are often expressed with words
\textbf{м\'{о}жно} and \textbf{нельз\'{я}}.

\begin{itemize}
\tightlist
\item
  Здесь м\'{о}жно жить. = One may live here.
\item
  Здесь нельз\'{я} есть. = One cannot/should not eat here.
\end{itemize}

The English translation may vary. You can specify the person for whom
the permission or recommendation applies, in the Dative (but you do not
have to):

\begin{itemize}
\tightlist
\item
  Мне нельз\'{я} спать. = I should not sleep.
\item
  Нам нельз\'{я} мн\'{о}го есть. = We should not eat a lot.
\end{itemize}

P.S. the \textbf{-ся} at the end of ``нр\'{а}вится'' is a reflexive particle
and comes after the ending (in verbs, use \emph{-сь} after a vowel,
\emph{-ся} after a consonant) . Technically, a reflexive verb is one
where the subject of the verb acts on itself. As you can see, often this
is not always reflected clearly in the meaning. «Нр\'{а}виться» is one of
those verbs that are reflexive ``just because''.

Don't worry about it too much for now, as we'll be tackling reflexives
in more detail further down the tree.
