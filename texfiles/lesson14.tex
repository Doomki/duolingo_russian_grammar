\chapter{Food}\label{food}

\section{Yummy}\label{yummy}

\begin{quote}
«для»(for) always takes Genitive nouns
\end{quote}

Food offers a delicious intake of mass nouns. Russian has them massed up
even where English does not!

\begin{itemize}
\tightlist
\item
  so \textbf{карт\'{о}шка}(potatoes), \textbf{лук}(onions), \textbf{шокол\'{а}д}
  are mass nouns in Russian
\item
  and you may recall that mass nouns may be used in Gen. instead of Acc.
  if you mean ``some quantity'': \emph{Куп\'{и} с\'{ы}ра/карт\'{о}шки.} = Buy some
  cheese/potatoes.
\end{itemize}

\section{'Taters}\label{taters}

The formal word for potato is \textbf{карт\'{о}фель} (German speakers,
rejoice), but it's hardly ever used in speech. Use «карт\'{о}шка» instead.

The word for tomato is \textbf{помид\'{о}р}. There is also the word
\textbf{том\'{а}т}, but it is

\begin{itemize}
\item
  the plant's name, pretty formal; look on pricetags
\item
  the base stem for derivative products: \emph{том\'{а}тная п\'{а}ста = tomato
  paste}
\item
  \textbf{посуда} is a word for different containers used for cooking ,
  consuming and further storage of food. English, sadly, does not have
  the exact equivalent. However, it is obviously ``dishes'' that you
  wash and ``cookware/tableware'' that you buy.
\end{itemize}

\section{Verbal wisdom}\label{verbal-wisdom}

In this skill, we used \emph{perfective} verbs for ``cook'', ``cut'',
``wash''. The reason is simple: that's the verb you'd use when you want
a \emph{single specific} action, often with a result---rather than
referring to ``activity'' (activity may be fun but, in some cases,
pointless).

More on that later. For now, just go with the flow.
