\chapter{Adjectives Basics and
Spelling}\label{adjectives-basics-and-spelling}

In Russian adjectives \emph{agree} with the nouns they modify in
gender/number and case. Out of 24 combinations only 12 forms are
different. This system is completely regular, with no change of stress.
The endings have ``hard'' and ``soft'' variants depending on the stem
(for example, ый/ий or ``ая/яя'').

\textbf{Here is the Nominative and Genitive} for ``classic'' hard- and
soft-stem adjectives (``new''/``blue''):

\begin{longtable}[]{@{}lll@{}}
\toprule
\begin{minipage}[b]{0.32\columnwidth}\raggedright\strut
\strut
\end{minipage} & \begin{minipage}[b]{0.32\columnwidth}\raggedright\strut
NOM\strut
\end{minipage} & \begin{minipage}[b]{0.32\columnwidth}\raggedright\strut
GEN\strut
\end{minipage}\tabularnewline
\midrule
\endhead
fem & н\'{о}в\textbf{ая}/с\'{и}н\textbf{яя} чашка &
н\'{о}в\textbf{ой}/с\'{и}н\textbf{ей} ч`{а}шки\tabularnewline
masc & н\'{о}в\textbf{ый}/с\'{и}н\textbf{ий} дом &
н\'{о}в\textbf{ого}/с\'{и}н\textbf{его} д\'{о}ма\tabularnewline
neut & н\'{о}в\textbf{ое}/с\'{и}н\textbf{ее} окн\'{о} &
н\'{о}в\textbf{ого}/с\'{и}н\textbf{его} окна\tabularnewline
pl. & н\'{о}в\textbf{ые}/с\'{и}н\textbf{ие} ч\'{а}шки &
н\'{о}в\textbf{ых}/с\'{и}н\textbf{их} ч\'{а}шек\tabularnewline
\bottomrule
\end{longtable}

note that \emph{masculine} and \emph{neuter} merge in all their forms
different from the Nominative one (their Accusative will be the same as
the Gen. or the Nom. depending on animacy). In the Nominative there is
also -ОЙ masculine ending: \textbf{больш\'{о}й} (``big''). Only for
ending-stressed adjectives.

\begin{itemize}
\tightlist
\item
  ОГО/ЕГО are historical spellings: \textbf{г} actually sounds like
  {[}\textbf{в}{]}
\item
  unstressed -ая(яя) /-ое (ее) sound identical in standard Russian:
  \emph{с\'{и}няя} and \emph{с\'{и}нее} have no difference in pronunciation.
\end{itemize}

The following universal rules of Russian spelling will give you the rest
of the endings for any adjective you ever meet (there exist 4 patterns
at most):

\begin{itemize}
\tightlist
\item
  After \textbf{Г-К-Х} (``velars'') and \textbf{Ш-Щ-Ж-Ч} (``hushes'')
  use \textbf{И} and never \emph{Ы}
\item
  After \textbf{Ц , Г-К-Х} (``velars'') and \textbf{Ш-Щ-Ж-Ч}
  (``hushes'') use \textbf{А, У} and never \emph{Я, Ю}
\item
  After \textbf{Ц} and \textbf{Ш-Щ-Ж-Ч} (``hushes'') use \textbf{Е}
  \emph{when unstressed} and never \emph{О}.
\end{itemize}
