\chapter{Conjunctions}\label{conjunctions}

\section{а vs. и}\label{ux430-vs.-ux438}

In Russian, \textbf{и} is used to show similarity. Otherwise you should
use \textbf{а}, which shows \emph{contrast}. To be more specific, here
are the typical patterns:

\begin{itemize}
\tightlist
\item
  Я мальчик, а ты девочка. = I am a boy and you are a girl.
\item
  Я работаю в кафе, а ты в школе. = I work in a cafe, and you (work) in
  a school.
\item
  Я люблю спать, а ты нет. = I like sleeping, and you don't.
\item
  А ты? = And you? $ \rightarrow$ often used to indicate a question.
\end{itemize}

\section{\texorpdfstring{зато (\emph{negative}, зато
\emph{positive})}{зато (negative, зато positive)}}\label{ux437ux430ux442ux43e-negative-ux437ux430ux442ux43e-positive}

A conjunction used for ``compensating'' for something unpleasant with
something that, you imply, is good:

\begin{itemize}
\tightlist
\item
  У нас нет молока, зато есть хлеб = \emph{We don't have milk but we do
  have bread}.
\item
  Мальчик ещё не умеет писать, зато хорошо читает. = \emph{The boy
  cannot write yet but he reads well}.
\end{itemize}

Not exactly the best thing to translate into English (\emph{``on the
other hand''? ``but at least''? ``thankfully?''}), so it is not often
used in this course.

\section{хотя ('though')}\label{ux445ux43eux442ux44f-though}

Much like the English \emph{though/even though/although}. It is often
combined with ``и'' before the predicate (which is sometimes directly
after «хотя»):

\begin{itemize}
\tightlist
\item
  Он здесь, хотя (он) и не знает ничего.= Он здесь, хотя (он) ничего и
  не знает. = He is here, even though he doesn't know anything.
\end{itemize}

\section{как}\label{ux43aux430ux43a}

This conjunction has a rather interesting use, to show when someone
perceives someone else's action:

\begin{itemize}
\tightlist
\item
  Я в\'{и}жу, как он\'{а} танц\'{у}ет. = I see her dancing.
\item
  Он\'{и} сл\'{у}шают, как музык\'{а}нт игр\'{а}ет. = They listen to the musician
  playing.
\end{itemize}

\begin{center}\rule{0.5\linewidth}{\linethickness}\end{center}

\emph{For \textbf{а}, there is also ``narrative'' contrast pattern,
largely absent from this course (but not from real-life Russian):}

\begin{itemize}
\tightlist
\item
  На столе чашка, а в чашке чай. = \emph{There is a cup on the table,
  and the cup has tea in it}.
\item
  Он здесь, а это значит --- воды нет. = \emph{He is here, and that
  means there's no water}.
\item
  Такси --- это машина, а машины не всегда хорошо работают. = \emph{A
  taxi is a car, and cars do not always work well.} (here, you are
  making your point by introducing a new thought ``unexpected'' by a
  listener)
\end{itemize}
