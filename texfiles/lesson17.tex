\chapter{There is}\label{there-is}

\section{Word order}\label{word-order}

To say ``there is/are'' in Russian, do the following:

\begin{itemize}
\tightlist
\item
  say THE PLACE
\item
  then the verb (if any)
\item
  then THE OBJECT
\end{itemize}

«есть» is not used, unless the sentence really has to emphasize the
existence of the object. Some examples:

\begin{itemize}
\tightlist
\item
  На стол\'{е} л\'{о}жка. = There is a spoon on the table.
\item
  На ст\'{у}ле м\'{а}льчик. = There is a boy on the chair.
\item
  В д\'{о}ме никог\'{о} нет. = There is no one in the house.
\item
  На стол\'{е} леж\'{и}т к\'{о}шка. = A cat is lying on the table.
\end{itemize}

In the Present tense no verb is necessary; in the past, you would at
least need a form of ``to be''. Note that even in the present Russian
still uses verbs like ``is situated'', ``stands'', ``lies'' way more
often than would be considered normal in English.

The most natural translation into English is a structure like ``There is
an apple on the table'' or ``An apple is on the table''. The emphasis is
on the object, not on the place.

\emph{Actually, such a sentence answers the question of WHAT is in the
said place. For out-of-the-blue sentences about objects that have
nothing unique about them it matches what English THERE-IS sentences are
for. So this is what we have in this course.}

\section{Actions}\label{actions}

The initial position of a ``place'' inside the sentence holds for many
other structures, too. Whenever the place is not a part of the
``message'' of your sentence, it is usually somewhere at the beginning
(that is, if the place frames your description of an action rather than
providing crucial information).

If the whole point of uttering a sentence is telling someone about the
place then, naturally, it takes the sentence-final position:

\begin{itemize}
\tightlist
\item
  З\'{а}втра я в Нью-Й\'{о}рке. = I am in New York tomorrow. (not somewhere
  else)
\end{itemize}

\section{lies/stands}\label{liesstands}

You don't have to translate verbs like ``to stand'' and ``to lie''
literally when they refer to objects. Such use is not, by a wide margin,
nearly as standard in English as it is in Russian:

\begin{itemize}
\tightlist
\item
  На стол\'{е} сто\'{и}т ч\'{а}шка. = \emph{A cup is (``stands'') on the table.}
\end{itemize}

In English ``to be'' is perfectly fine, so we accept that.
