\chapter{Questions}\label{questions}

\section{Where}\label{where}

Russian makes a distinction between being somewhere (тут/здесь, там) ,
going there (сюда, туда) and coming \emph{from} there (отсюда,
оттуда)---so naturally question words follow suit:

\begin{itemize}
\tightlist
\item
  Где? = Where (at)?
\item
  Куда? = Where to?
\item
  Откуда? = Where from?
\end{itemize}

\section{What or what are you?}\label{what-or-what-are-you}

Russian uses \textbf{«Кто»}(who) when asking about identity and
occupation and «Что» is used for objects rather than people. Since
Russian nouns have cases, \textbf{кто} and \textbf{что} also change
depending on their role in the implied sentence. As you will discover a
little bit further down the tree, «Кто» behaves rather like a masculine
adjective.

\begin{longtable}[]{@{}llll@{}}
\toprule
CASE & What & Who & Whose\tabularnewline
\midrule
\endhead
Nom. & что & кто & чей, чьё, чья ,чьи\tabularnewline
Gen. & чег\'{о} & ког\'{о} & чьег\'{о}, чьег\'{о}, чь\'{е}й, чьих\tabularnewline
Acc. & что & кого & Gen/Nom; «чью» for Fem.\tabularnewline
Prep & чём & ком & чьём, чьём, чьей, чьих\tabularnewline
\bottomrule
\end{longtable}

\section{Почему? and
Зачем?}\label{ux43fux43eux447ux435ux43cux443-and-ux437ux430ux447ux435ux43c}

\begin{itemize}
\tightlist
\item
  \textbf{Почему} is used when asking a question about a cause of some
  event or action. It is a question that looks back at the past.
\item
  \textbf{Зачем} starts a question about the \textbf{purpose} of some
  action or some event that can have one. It is a question that looks
  towards a desired future.
\end{itemize}

In a few regions of Russia (Tatarstan, for example) people may use
\emph{зачем} for both questions if their usage of Russian is influenced
by a major local language that makes no distinction between the two. In
Standard Russian these are two clearly separate entities.
