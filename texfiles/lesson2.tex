\chapter{Phrases 1}\label{phrases-1}

\section{Hello}\label{hello}

Russian has a more informal greeting \textbf{«Прив\'{е}т»} and a more formal
\textbf{«Здр\'{а}вствуй(те)»}. Here, we focus on the first, since it is the
shorter one.

When on phone, use \textbf{«Алл\'{о}» (Алё)}.

\textbf{«Пож\'{а}луйста»} (please) has another popular position in the
sentence---namely, after the verb (more on that later).

• \emph{you can also use} «пож\'{а}луйста» \emph{as a reply to ``thanks'',
meaning ``You are welcome!''}

\section{How are you?}\label{how-are-you}

The phrase for ``How are you?'' literally means ``How are your affairs
(the stuff you do)?''

No one uses it as a greeting, i.e. you are not expected to use it with
people you barely know (or those you know, for that matter). And be
prepared for a person to \emph{actually} tell you how they've been
doing. ;)

\section{Good morning!}\label{good-morning}

Morning typically starts at 4 or 5 a.m., afternoon at noon, evening at 5
p.m. (at 6 for some) and night at 11 or at midnight.

You only use ``Good night'' (Спок\'{о}йной н\'{о}чи) when parting before sleep
(or saying your goodbyes really late, so it is implied you or the
listener are going to bed soon after).

\emph{If you are advanced enough to have noticed oblique forms used in
some phrases---you are right! Greetings and other similar expressions
are often shortened versions of longer phrases, where words still retain
their forms. For example,} «Спок\'{о}йной н\'{о}чи» \emph{probably replaces the
longer} «Я жел\'{а}ю вам спок\'{о}йной н\'{о}чи!» \emph{(I wish you a peaceful
night). Needless to say, the full version is never used.}
