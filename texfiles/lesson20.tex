\chapter{Around you}\label{around-you}

\section{Do that the English way}\label{do-that-the-english-way}

To express the idea of speaking some language, or something being
written in that language, Russian has adverbs literally meaning
``Russian-ly'', ``English-ly'' etc.. :

\begin{itemize}
\tightlist
\item
  Я не говор\'{ю} по-р\'{у}сски. = I do not speak Russian.
\item
  Вы говор\'{и}те по-англ\'{и}йски. = Do you speak English?
\end{itemize}

They are formed from \textbf{-ский} adjectives by attaching \textbf{по-}
and changing the tail to bare \textbf{-ски}: по-р\'{у}сски, по-италь\'{я}нски,
по-яп\'{о}нски, по-вьетн\'{а}мски, по-америк\'{а}нски, по-франц\'{у}зски and so on.

And remember, these words actually mean something done ``in a certain
way'', so «суши по-американски» (American-style sushi) should not
surprise you!

\section{Locative 2}\label{locative-2}

A relatively small group of short masculine nouns have an accented
\textbf{-у} ending with \emph{в/на} in the meaning of place (and only
then):

\begin{itemize}
\tightlist
\item
  Мы в \textbf{аэропорт\'{у}}. = We are at the airport.
\item
  Я спл\'{ю} на \textbf{пол\'{у}}. = I sleep on the floor.
\end{itemize}

Our course has about a dozen of them (there are about 100 in the
language). Also, there exists a very small group of feminine nouns, all
``-ь''-ending, that have a stressed Locative-2 ending:

\begin{itemize}
\tightlist
\item
  Твой св\'{и}тер в \textbf{кров\'{и}}. = Your sweater is covered in blood.
\end{itemize}

All these nouns use their normal Prepositional form with ``о'' and
``при''.

\paragraph{совс\'{е}м}\label{ux441ux43eux432ux441ux435ux43c}

This word is used with qualities that manifest ``totally''--- usually
with negatives:

\begin{itemize}
\tightlist
\item
  Он совс\'{е}м не раб\'{о}тает. = He doesn't work at all.
\item
  Том совс\'{е}м не ест. = Tom doesn't eat at all.
\item
  Мы совс\'{е}м бл\'{и}зко. = We are really close (i.e. almost there).
\end{itemize}

\paragraph{междунар\'{о}дный}\label{ux43cux435ux436ux434ux443ux43dux430ux440ux43eux434ux43dux44bux439}

It comes from «м\'{е}жду» + «нар\'{о}ды», i.e. ``between''+``peoples'', which is
quite literally ``international''.

The loanword «интернациональный» means the same but has quite limited
use in certain combinations like ``international team'' or
``international debt'' (mostly these are from political contexts). This
course largely avoids this word.

Probably, ``international team/orchestra'' etc. is the context where you
must use «интернациональный»).

\paragraph{жив\'{о}тное}\label{ux436ux438ux432ux43eux442ux43dux43eux435}

The word for an ``animal'' is a nominalised neuter adjective, and its
case forms follow adjectival pattern. Of course, its gender is fixed:

\begin{itemize}
\tightlist
\item
  Это жив\'{о}тные. = These are animals.
\item
  Я любл\'{ю} жив\'{о}тных. = I like animals.
\end{itemize}
