\chapter{Prepositions and Places}\label{prepositions-and-places}

\section{Verbs of motion}\label{verbs-of-motion}

Russian distinguishes between ``going'' on foot and by some sort of
vehicle. If you aren't moving within the city, use a `vehicle verb'
\textbf{ехать} (one-way movement) or \textbf{ездить} (repeated, round
trip or in general). More on that later, in ``Motion verbs''.

\section{Into/onto\ldots{} at-to?}\label{intoonto-at-to}

Once again, with \textbf{в} and \textbf{на} you use Prepositional for
\emph{location}, and Accusative for \emph{direction}:

\begin{itemize}
\tightlist
\item
  Я жив\'{у} в \emph{Л\'{о}ндоне}. \textasciitilde{} I live in London.
\item
  Я \'{е}ду в \emph{Л\'{о}ндон}. \textasciitilde{} I am going (by vehicle) to
  London.
\end{itemize}

Here is a `cheat sheet' of forms you'll need for places (no living
beings, so---the easy Accusative for masculine):

\begin{longtable}[]{@{}llll@{}}
\toprule
Nominative & Acc. & Prep. & example\tabularnewline
\midrule
\endhead
-а/-я & -у/-ю & -е & Америка $ \rightarrow$ в Америку/в Америке\tabularnewline
\o/-о/-е & \o/-о/-е & -е & стол $ \rightarrow$ на стол / на столе\tabularnewline
-ь \emph{feminine} & -ь & -и & дверь $ \rightarrow$ на дверь/на двери\tabularnewline
\textbf{-ия} & -ию & -ии & Англия $ \rightarrow$ в Англию/в Англии\tabularnewline
\textbf{-ие} & -ие & -ии & здание $ \rightarrow$ в здание / в здании\tabularnewline
\bottomrule
\end{longtable}

\section{Word choice}\label{word-choice}

For ``outdoors'' Russians use \textbf{«на улице»} (literally, ``on the
street'').

The preposition \textbf{о} (об) means ``about'' only as in the sense of
``thinking/writing about''. Don't use it for ``approximately''. With
«\emph{мне}» a special form is used, \textbf{обо}.

The contraction ``USA'' or ``the U.S.'' is \textbf{США} (сэ-шэ-А, with
the stress on the last vowel).

There is no difference drawn between ``city'' and ``town''.

In Russian it is typical to describe objects as ``standing'', ``lying'',
``being situated'', ``hanging''. This is rare in English, and often
sounds unnatural, therefore in this course it is perfectly OK to
translate a ``whereabouts''-verb with a simple ``is'', ``was'' etc.

\section{Here and there, and here}\label{here-and-there-and-here}

For ``here'', the words \textbf{здесь} and \textbf{тут} are almost
completely interchangeable in any imaginable context. \textbf{Тут} is
considered a bit more informal, and is used in set expressions ( тут
же\textasciitilde{}right away, тут и там). «Здесь» is somewhat less
suitable for figurative meanings (when by ``here'' you mean the current
situation rather than a place). In this course, they are completely
interchangeable when not being used in a set expression.

\textbf{находиться} is a verb to denote the whereabouts of things, and,
sometimes of people (when the emphasis is on exactly where they are). It
could be translated as ``to be situated'' or ``to be located'', but as
these verbs usually sound rather formal in English, so you can just use
``to be''.

\textbf{около} is almost the same as «возле». It can also be used in the
sense of ``about''(=approximately).
