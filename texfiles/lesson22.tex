\chapter{Verbs Present 2}\label{verbs-present-2}

\section{$ \rightarrow$ and $ \rightleftharpoons$}\label{and}

There are two options for verbs of \textbf{going}: a specific
1-directional verb and also repeated motion, multi-directional verb. For
now, stick to this rule for \textbf{идт\'{и} / ход\'{и}ть}:

\begin{longtable}[]{@{}lll@{}}
\toprule
\begin{minipage}[b]{0.32\columnwidth}\raggedright\strut
\strut
\end{minipage} & \begin{minipage}[b]{0.32\columnwidth}\raggedright\strut
ENG\strut
\end{minipage} & \begin{minipage}[b]{0.32\columnwidth}\raggedright\strut
RUS\strut
\end{minipage}\tabularnewline
\midrule
\endhead
\textbf{right now}$ \rightarrow$ & I \emph{am going}. & Я
\textbf{ид\'{у}}.$ \rightarrow$\tabularnewline
\emph{habitual} $ \rightleftharpoons$ & I often \emph{go} there. & Я ч\'{а}сто туд\'{а}
\textbf{хож\'{у}}.\tabularnewline
\emph{generic} $ \leadsto$$ \curvearrowleft$$ \circlearrowleft$ & The baby already \emph{walks}. I~\emph{am walking}
(around). & Ребёнок уж\'{е} \textbf{х\'{о}дит}. Я \textbf{хож\'{у}}.\tabularnewline
\bottomrule
\end{longtable}

\section{Asking}\label{asking}

\begin{itemize}
\tightlist
\item
  \textbf{прос\'{и}ть} $ \rightarrow$ to ask for/beg for/request something
\item
  \textbf{спр\'{а}шивать} $ \rightarrow$ to ask a question (i.e. ask for information)
\end{itemize}

In other words, when using \emph{пр\'{о}сит}, one wants to be given
something (or for something to be done). He~who \emph{спр\'{а}шивает}--wants
an answer.

By the way, ``to ask a question'' is, actually, «\emph{зад\'{а}ть/задав\'{а}ть}
вопр\'{о}с». Those who speak German may recall \emph{eine Frage stellen},
which works in a similar way (apparently, ``to ask an asking'' is no
good in German, either).

\section{Negative sentences}\label{negative-sentences}

Remember that Russian sort of uses double and triple negatives. To be
more precise, it is coordinated negation: when the sentence is negative,
you should automatically negate every pronoun referring to
\emph{someone}, \emph{anywhere}, \emph{some time}, \emph{anything},
\emph{in some way} and so on:

\begin{itemize}
\tightlist
\item
  Мы никогд\'{а} ник\'{у}да ни с кем не ход\'{и}м = We never go anywhere with anyone
  (Literally, ``We never to nowhere with nobody don't go'').
\end{itemize}

They all change to \emph{nobody, nowhere, never, nothing, by no means}
and so on. \emph{No one} and \emph{nothing} will have the correct case
(though, ``nothing'' is virtually always \emph{ничего}, not
\emph{ничто})

\section{Adverbs}\label{adverbs}

The typical position for \textbf{-о}(-е)-ending adverbs is \emph{before}
the verb. For example:

\begin{itemize}
\tightlist
\item
  «Он хорош\'{о} вид\'{и}т»=``He sees well''.
\item
  «Том б\'{ы}стро ушёл»=``Tom left quickly''
\end{itemize}

\section{Consonant mutation.}\label{consonant-mutation.}

You might have noticed that the consonant before the ending is sometimes
different in the infinitive than in the personal forms. It is called
\emph{mutation} and is quite similar to the process that makes ``tense''
into ``tension'' (where an ``s'' turns into a ``sh''). Here are the
patterns you might encounter:

\begin{itemize}
\tightlist
\item
  Б, П, В, Ф, М adding \textbf{л} (люб\'{и}ть / любл\'{ю})
\item
  С, Х becoming \textbf{ш} (пис\'{а}ть / пиш\'{у})
\item
  Д, З becoming \textbf{ж} (в\'{и}деть / в\'{и}жу)
\item
  Т, К becoming \textbf{ч} (плат\'{и}ть / плач\'{у})
\item
  СК, СТ becoming Щ (пуст\'{и}ть / пущ\'{у})
\end{itemize}

If there is alteration, there is a rule:

\begin{itemize}
\tightlist
\item
  \textbf{И-conjugation verbs} only have 1st person singular mutated. It
  is normal (e.g. люблю / любит)
\item
  \textbf{Е-conjugation verbs} have mutation in ALL personal forms (if
  any). It is \emph{non-productive behaviour}, which in practice means
  that a lot of popular verb stems still have this behavior (e.g.
  писать, сказать). However, new Е-conjugation verbs do not get this
  pattern.
\end{itemize}

рисовать $ \rightarrow$ рисую, on the other hand, is a regular transformation of
-овать/-евать verbs

\section{Playing}\label{playing}

The verb «играть» is used as follows:

\begin{itemize}
\tightlist
\item
  \textbf{в} + Accusative for games
\item
  \textbf{на} + Prepositional for musical instruments
\end{itemize}

For example, Я игр\'{а}ю в футб\'{о}л / Я игр\'{а}ю на гит\'{а}ре.
