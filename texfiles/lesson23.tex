\chapter{Time and Numbers}\label{time-and-numbers}

\section{Using numbers with nouns}\label{using-numbers-with-nouns}

Let's focus on the Nominative for now (this also works when Acc.=Nom).
Russian numbers may seem a bit weird. The case of the noun depends on
the last word of the number:

\begin{longtable}[]{@{}llll@{}}
\toprule
last word & means & Case & example\tabularnewline
\midrule
\endhead
од\'{и}н \emph{(одн\'{а}, одн\'{о}, одн\'{и} )} & 1 & Nom. sg. & од\'{и}н дом, одн\'{а} м\'{а}ма,
дв\'{а}дцать одн\'{о} окн\'{о}, од\'{и}н ст\'{о}л\tabularnewline
два \emph{(две)}, три, чет\'{ы}ре & 2, 3, 4 & Gen. sg. & две к\'{о}шки, два
стол\'{а}, три м\'{а}льчика, тр\'{и}дцать чет\'{ы}ре стол\'{а}\tabularnewline
Larger than that & 5, 6, 12, 100 etc. & Gen. pl. & пять к\'{о}шек, пять
м\'{а}льчиков, дв\'{а}дцать пять к\'{о}шек, милли\'{о}н к\'{о}шек\tabularnewline
\bottomrule
\end{longtable}

Just like English, Russian has words for \emph{eleven through nineteen},
so they fall into the ``larger'' category.

\textbf{Genitive plural} has a rather bizarre set of patterns, so a
separate skill later on will teach you how to make it for most nouns.

\section{Expressions}\label{expressions}

\begin{itemize}
\tightlist
\item
  \emph{I am 10 (years old)} = Мне д\'{е}сять (лет) Note the Dative ``мне''
  and the number in the Nominative. The Genitive plural ``лет'' is
  irregular.
\end{itemize}

The Dative forms of он, она and они are ему, ей, им respectively.

\begin{itemize}
\tightlist
\item
  \emph{at 9 o'clock} = в д\'{е}вять час\'{о}в $ \rightarrow$ the Accusative here (same as
  the Nominative)
\item
  \emph{at 2 in the morning} = в два (час\'{а}) н\'{о}чи (in Russian `morning'
  starts at about 4-5 a.m.)
\item
  \emph{in January, June. etc.} = в январ\'{е}, и\'{ю}не \ldots{}
  (Prepositional). Note that all the month names are masculine nouns.
\item
  the beginning/end of July = нач\'{а}ло/кон\'{е}ц и\'{ю}ля
\end{itemize}

\begin{center}\rule{0.5\linewidth}{\linethickness}\end{center}

Why are Russian numbers so strange? Well, for 2-3-4 these are the
remnants of Dual number (which is between the singular and the plural).
As for the larger numbers, they are essentially ``nouns'': \emph{a heap
of cats, a lot of cats, a thousand\ldots{} of cats}.

\section{Сейчас}\label{ux441ux435ux439ux447ux430ux441}

Russian uses two words for ``now''. One is «сейчас», which means ``now,
at the moment'', and describes the current moment in a neutral manner,
often implying that things change and the state described is attributed
to this particular moment. It can change soon:

\begin{itemize}
\tightlist
\item
  Сейчас никого нет дома. = No one is home (right) now.
\item
  Сейчас пять утра. = It is 5 a.m. now.
\end{itemize}

Теперь is the ``now'' you use when things are different from ``before''.
You imply that the situation has changed. It is also associated with a
more prolonged period of time, i.e. the state of affairs is different
from before, and will stay so for now:

\begin{itemize}
\tightlist
\item
  Мы теперь работаем в главном офисе. = We now work at the main office.
  \emph{(We did not, but now we are, and things are going to stay like
  that for some time)}
\end{itemize}

\section{время}\label{ux432ux440ux435ux43cux44f}

The noun «время» (``time'') belongs to a really small class of neuter
nouns. Its Genitive form is времени, and all other oblique forms also
retain the -ен part.
