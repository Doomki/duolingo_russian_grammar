\chapter{Family}\label{family}

Not much to say here, except that Russian does not have a special word
for \emph{siblings} or \emph{grandparents}.

Unlike English, Russians rarely say ``my mother'', ``my grandfather'';
usually they omit ``my''.

\section{\texorpdfstring{\textbf{свой} \textasciitilde{} one's
own}{свой \textasciitilde{} one's own}}\label{ux441ux432ux43eux439-ones-own}

\ldots{}And when they don't, it is more natural to use reflexive
``свой'' (one's own). English does not have anything quite like that.
Essentially, it is a substitute for \emph{my, your, his, her} etc. that
you use when it refers to the person (or thing) that is the subject of
the sentence or, at least, the clause you are in. A few typical
examples:

\begin{itemize}
\tightlist
\item
  Кошка ест из \emph{своей} миски = The cat is eating out of \emph{its}
  bowl.
\item
  Мы у \emph{(своих)} родителей = We are at \emph{our} parents' place.
  (here you can omit ``своих'')
\item
  Я думаю, он у \emph{своих} родителей = I think he's at \emph{his}
  parents' place.
\end{itemize}

Forms of «свой» follow the same mostly-adjectival pattern that
«мой»,«твой», «ваш», «наш» and «этот» use: свой, своя, своё, свои $ \rightarrow$
своего, свою, своих\ldots{}

Since «свой» describes something belonging to the subject of the
sentence, it cannot be used with the subject of the sentence itself. The
exception is made when you are making generalisations, e.g. ``One's
(own) reputation is always more important''\textasciitilde{}«Своя
репутация всегда важнее».

Pay attention to what the \emph{grammatical subject} is. Sentences like
«Мне нравится у своей сестры» are sort-of-OK sometimes, but you are
really treading on thin ice here. This one sounds almost normal, while
some others would immediately look unnatural.

\section{Mister!}\label{mister}

In spoken Russian «дядя»(uncle) and «тётя»(aunt) are often used to refer
to some adult ``guy'' or ``woman''. A special case is children's use,
since they often use it even as a form of address (``тётя Маша!'').
