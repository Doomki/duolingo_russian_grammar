\chapter{People 2}\label{people-2}

\section{A student}\label{a-student}

Russian has different words for a school student (aka \emph{a pupil},
BrE) and a college-level student, which both have masculine and feminine
versions:

\begin{itemize}
\tightlist
\item
  \textbf{учен\'{и}к / учен\'{и}ца} -- a school student or a student/apprentice
  in general, especially in spiritual sense
\item
  \textbf{студ\'{е}нт / студ\'{е}нтка} -- a college or university-level student
  (attends a corresponding institution)
\end{itemize}

\textbf{Молод\'{е}ц} is a word you use when someone ``did a good job''. It
comes with a patronizing shade, so ideally you use it towards your
friends or actual students/ subordinates (but not towards people whose
work you are in no position to judge).

\section{5 men}\label{men}

When you are counting people, use ``челов\'{е}к'' for numbers that end in
«пять» (5) or more. Anywhere else use the normal Genitive plural
``люд\'{е}й'' (with \emph{много} and \emph{мало} both are possible, but I'd
stick to \emph{люд\'{е}й}).

\section{Learning and studying}\label{learning-and-studying}

OK, Russian has a number of ways to express learning, but in this course
we have \textbf{уч\'{и}ться}, \textbf{уч\'{и}ть}, and \textbf{заним\'{а}ться}. The
1st verb, \textbf{уч\'{и}ться}, is introduced in this skill. Here is a bit
more:

\begin{longtable}[]{@{}lll@{}}
\toprule
\begin{minipage}[b]{0.32\columnwidth}\raggedright\strut
\strut
\end{minipage} & \begin{minipage}[b]{0.32\columnwidth}\raggedright\strut
meaning\strut
\end{minipage} & \begin{minipage}[b]{0.32\columnwidth}\raggedright\strut
examples\strut
\end{minipage}\tabularnewline
\midrule
\endhead
\textbf{уч\'{и}ться} & to study (e.g. to attend classes or to do self-study)
& Днём я уч\'{у}сь.\tabularnewline
\textbf{уч\'{и}ться в}(на) + Prep. & to study somewhere; to be in \emph{nth}
grade/\emph{nth} year & Д\'{е}вочка \'{у}чится в шк\'{о}ле, в 3-м
кл\'{а}ссе.\tabularnewline
уч\'{и}ть + Acc.\emph{subject} & to learn, to memorize something («наиз\'{у}сть»
=``by heart'') & Я уч\'{у} слов\'{а}. Я уч\'{у} р\'{у}сский яз\'{ы}к.\tabularnewline
уч\'{и}ть + Acc + Dat & to teach somebody something & Я уч\'{у} студ\'{е}нтов
р\'{у}сскому.\tabularnewline
\bottomrule
\end{longtable}

\section{Doctor}\label{doctor}

The usual word for a (medical) doctor is «\textbf{врач}». Then you have
«д\'{о}ктор», which is also OK but informal. However, a ``doctor'' as a
person with this level of post-graduate qualification is «д\'{о}ктор» with
no alternatives.
