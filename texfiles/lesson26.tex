\chapter{Weather and Nature}\label{weather-and-nature}

\section{It's raining}\label{its-raining}

``To go'' is the verb used for precipitation in Russian:

\begin{itemize}
\tightlist
\item
  Идёт дождь = It is raining.
\item
  Идёт снег. = It is snowing.
\item
  Идёт град. = it is hailing \emph{(we don't have it in the course)}.
\end{itemize}

\section{in summer/winter}\label{in-summerwinter}

Russian has adverbs for ``in spring'', ``in summer'' etc. They are
formed as the \emph{Instrumental} case of a corresponding noun.

We'll cover Intrumental in detail later. Right now just get used to the
words themselves:

\begin{itemize}
\tightlist
\item
  Весн\'{о}й м\'{о}кро. = It's wet in spring
\item
  Зим\'{о}й хорош\'{о} = It's good/nice in winter.
\item
  \'{О}сенью гр\'{я}зно. = It's muddy in the fall.
\item
  Л\'{е}том с\'{о}лнечно. = It is sunny in summer.
\end{itemize}

Russians usually assign each season 3 months, i.e. winter is December
through February and spring is March through May (even if you have snow
well into April).

\section{Category of State}\label{category-of-state}

It is easier than it sounds. When expressing a ``state'', some modality,
or one's opinion on the situation, Russian often uses these
\emph{impersonal} words, saying that such and such state is observed:

\begin{itemize}
\tightlist
\item
  Мне \textbf{х\'{о}лодно}. = I'm (feeling) cold.
\item
  На \'{у}лице \textbf{тепл\'{о}}. = It is warm outside.
\item
  \textbf{Хорош\'{о}}, что вы тут. = It is good you are here.
\item
  \textbf{Тр\'{у}дно} сказ\'{а}ть. = It is hard to say.
\end{itemize}

Many are homonymous with adverbs and short-form adjectives. So we'll
study them later with adjectives. For now, we' only have a handful of
such words useful when discussing the weather.

Needless to say, these do not use any grammatical subject and are quite
useful with verbs like ``to be'' and ``to become'' (``It's getting
warmer'').

The concept of ``category of state'' is not even taught to native
speakers. However, it does have a distinctive pattern of use. Makes it
easier to learn when you know why you say «мне х\'{о}лодно».
