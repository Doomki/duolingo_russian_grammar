\chapter{Sport}\label{sport}

\section{win/lose}\label{winlose}

Here you encounter two \emph{perfective} verbs; these two very obviously
refer to a specific result:

\begin{itemize}
\tightlist
\item
  Ты проигр\'{а}л! = You lost!
\item
  Нам надо в\'{ы}играть. = We need to win.
\end{itemize}

Note the formation of the past. If you remember \emph{был, был\'{а}}---all
Russian past forms are essentially formed the same way. The endings
correspond to gender and number:

\begin{longtable}[]{@{}llll@{}}
\toprule
\textbf{masc} & \textbf{fem} & \textbf{neut} &
\textbf{pl}\tabularnewline
\midrule
\endhead
--- & \textbf{-а} & \textbf{-о} & \textbf{-и}\tabularnewline
\bottomrule
\end{longtable}

We'll be practising many more past forms in the skill in the next row.

\section{Reflexive}\label{reflexive}

As a reminder, if a verb has \textbf{-ся} at the end, you stick it after
the usual ending («-сь» is used after a vowel):

\begin{itemize}
\tightlist
\item
  кат\'{а}ться на л\'{ы}жах = to ski
\item
  Я кат\'{а}юсь на л\'{ы}жах = I ski.
\end{itemize}

\section{«беж\'{а}ть», to
run}\label{ux431ux435ux436ux430ux442ux44c-to-run}

In this skill, we introduce the one-way verb ``to run''. You may not
remember but it has one of the four irregular stems:

\begin{longtable}[]{@{}ll@{}}
\toprule
SING. & PLUR.\tabularnewline
\midrule
\endhead
Я \textbf{бег\'{у}} & Мы беж\'{и}м\tabularnewline
Ты беж\'{и}шь & Вы беж\'{и}те\tabularnewline
Он беж\'{и}т & Они \textbf{бег\'{у}т}\tabularnewline
\bottomrule
\end{longtable}
