\chapter{Adjectives 1}\label{adjectives-1}

In Russian the adjective \emph{agrees} with the noun it modifies in
gender(number) and case. Fortunately, the system is \emph{completely}
regular and the stress stays the same. The forms for the cases you know
are:

\begin{longtable}[]{@{}llll@{}}
\toprule
\textbf{ENDINGS} & masculine & neuter & feminine\tabularnewline
\midrule
\endhead
Nom. & \textbf{-ый(\'{о}й)/-ий} & \textbf{-ое/-ее} &
\textbf{-ая/-яя}\tabularnewline
Acc. & Nom. or Gen. & -ое/-ее & -ую/-юю\tabularnewline
Gen. & -ого/-его & see masc. & -ой/-ей\tabularnewline
Prep. & -ом/-ем & see masc. & -ой/-ей\tabularnewline
\bottomrule
\end{longtable}

The plural ending in the Nominative is \textbf{-ые (ие)}. We will
address the oblique forms later in the course.

(we are not teaching possessive adjectives for now, )

A few examples:

\begin{itemize}
\tightlist
\item
  Я живу в большом городе. (Prep.,masc.)
\item
  Дайте большого кота. (Acc.,masc.)
\item
  Нам надо найти хорошую книгу. (Acc.,fem)
\end{itemize}

\section{velars and hushes}\label{velars-and-hushes}

Adjectives with the stem on \textbf{-к, -г, -х, -ш, -щ, -ж, -ч} will use
``и'', ``а'', ``у'' instead of ``ы'', ``я'', ``ю'' so watch carefully
(``русский'', for instance).

We will tackle the endings one at a time.

\section{целый vs.
весь}\label{ux446ux435ux43bux44bux439-vs.-ux432ux435ux441ux44c}

In Russian the idea of ``the whole'' of something can be expressed by
either «целый» or «весь». The former is used when implying the
unexpectedly ``large'' amount; it is the one we're teaching in this
skill:

\begin{itemize}
\tightlist
\item
  Он целый день спит. (normally, a person should have been awake for a
  long time)
\end{itemize}
