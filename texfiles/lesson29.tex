\chapter{Past \& Infinitive}\label{past-infinitive}

\section{The infinitive stem}\label{the-infinitive-stem}

In Russian the Past tense and the Infinitive are formed from the same
stem.

The forms are actually much easier than in the Present because there are
only four forms in total for \emph{masculine/feminine/neuter + plural},
similar to adjectives (the forms were participles once).

\begin{longtable}[]{@{}lllll@{}}
\toprule
VERB & \textbf{masc} & \textbf{fem} & \textbf{neut} &
\textbf{pl}\tabularnewline
\midrule
\endhead
\textbf{ending} & --- & \textbf{-а} & \textbf{-о} &
\textbf{-и}\tabularnewline
быть & был & был\'{а} & б\'{ы}ло & б\'{ы}ли\tabularnewline
есть & ел & \'{е}ла & \'{е}ло & \'{е}ли\tabularnewline
\bottomrule
\end{longtable}

«идти» and all its derivatives (пойт\'{и}, прийт\'{и}, найт\'{и}..) has a strange,
irregular past stem:

\textbf{walked, went}: он пошёл, она пошл\textbf{\'{а}}, оно пошл\textbf{\'{о}},
они пошл\textbf{\'{и}}

For the masculine form, there is a phonetic simplification for verbs
with infinitives in \emph{-чь,-сти/-зти, -зть/-сть}. For example
``мочь''(``can''), ``ползт\'{и}''(crawl) and ``лезть''(climb): \emph{он
\textbf{мог, полз, лез}} --- no final \textbf{Л} here.

This skill mostly covers the past form of imperfective verbs (only
«уст\'{а}ть» and «подожд\'{а}ть» are perfective). What it means for you is that
when 2 or more such actions are mentioned, they were all happening at
the same time or in no particular order. Why? Imperfective verbs like
«идт\'{и}», «жить», «говор\'{и}ть» are by nature unspecific about their exact
time frame.

\begin{itemize}
\tightlist
\item
  they express repeated or prolonged action
\item
  they express action in progress
\item
  they can also express the fact that an action has or has not occured
  (with or without details on ``when'' it took place).
\end{itemize}

\section{What about the present
form?}\label{what-about-the-present-form}

For some verb types the two stems are nearly identical (поним\'{а}ть,
говор\'{и}ть). Which is a good thing for you!
