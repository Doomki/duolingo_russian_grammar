\chapter{Basics 2}\label{basics-2}

\section{\texorpdfstring{\emph{I \textbf{have} a
cat}}{I have a cat}}\label{i-have-a-cat}

English prefers to express ownership and ``possession'' with the verb
``have''. In Russian ``existence'' is almost universally used instead
(in the official/academic style «иметь» \emph{to have} is OK to use).

Use it like that:

\begin{itemize}
\tightlist
\item
  \textbf{У \emph{A} есть \emph{X}} \textasciitilde{} by A there is an X
  $ \rightarrow$ A has an X
\end{itemize}

The owner is in the \emph{Genitive case} (more on that later) while X is
formally the subject. For now we will only study the Genitive form for
some pronouns.

\section{\texorpdfstring{\emph{You \textbf{have} wonderful
eyes!}}{You have wonderful eyes!}}\label{you-have-wonderful-eyes}

\textbf{Omit ''есть''} if the \emph{existence} of the object is obvious
or not the point --- very typical for describing traits or a number of
objects (``Tom has a beautiful smile/large eyes'', ``She has a very fat
cat''). This also applies to expressing temporary states and illnesses
(``She has a migraine'').

\section{\texorpdfstring{\emph{I eat/ She
eat\textbf{s}}}{I eat/ She eats}}\label{i-eat-she-eats}

In English, the only way a verb changes in the present tense is that you
add -s for the 3rd person singular. In Russian, all 6 forms are
different and fit two regular patterns.

However, \textbf{\emph{eat}} «есть» and \textbf{\emph{want}} «хот\'{е}ть»
are two of the four verbs that are irregular (that is, do not strictly
follow any of the 2 patterns).

Note that the ``present'' tense is formed from one stem and the ``past''
and infinitive from the second one. In general, these two are slightly
different. For now, don't worry about the infinitive stem.

\section{Hard and soft}\label{hard-and-soft}

Russian consonants are split into two groups of 15, which are pronounced
in two different ways, \emph{palatalized} (aka ``soft'') and
\emph{non-palatalized} (aka ``hard''). We'll stick to the shorter
``soft'' and ``hard'' (sorry).

When a consonant is ``soft'' it means than you pronounce it with you
tongue raised high; for ``non-palatalized'' consonants it stays low.
Russian orthography has its history but, long story short, you can tell
the ``softness'' of a consonant from a vowel letter spelled afterwards:

\begin{itemize}
\tightlist
\item
  \textbf{А, Ы, У, Э, О} follow ``hard'' consonants
\item
  \textbf{Я, И, Ю, Е, Ё} follow palatalized ones
\end{itemize}

If there is nothing after a consonant, the soft sign Ь is used to show
the softness. In consonant clusters palatalization is predictable from
the softness of the last consonant. We aren't teaching it here. These
days the trend is to only ``soften'' the last consonant in most
clusters, while a hundred years ago some clusters were palatalized even
without any obvious reason.

To show you how it works, here is an example, using an ad-hoc
transcription:

\begin{itemize}
\tightlist
\item
  ж\'{е}\textbf{нщ}ина = {[}жэн$^j$щи$^e$на{]}
\item
  \textbf{ст}ен\'{а} = {[}ст$^j$и$^e$на{]} or {[}с$^j$т$^j$и$^e$на{]}
\end{itemize}

There are dictionaries («орфоэпический словарь») that show the
recommended pronunciation of words and contain general pronunciation
rules, too.

\section{Voicing}\label{voicing}

Some consonants let your voice come out immediately (voiced) while
others wait for the release of the consonant and only then let your
voice escape (unvoiced). In Russian there are 6 pairs of such
consonants: Б/П, В/Ф, Г/К, Д/Т, Ж/Ш, З/С.

\begin{itemize}
\tightlist
\item
  whenever one of these consonants (except В) follows another, the
  second overrides or reverses the voicing of the first: сд = {[}зд{]},
  вс= {[}фс{]}
\item
  the end of the phrase is unvoiced: этот клуб \emph{{[}клуп{]}}
\item
  rules apply between the word boundaries, too
\item
  Х, Ч, Ц, Щ also play this game, even though Russian lacks letters for
  their voiced partners ({[}$ \gamma${]}, {[}дж'{]}, {[}дз{]}, {[}ж'ж'{]}). They
  will devoice the preceding consonant or become voiced themselves.
\end{itemize}

Unlike Ukrainian, Russian only uses {[}$ \gamma${]}, {[}дж'{]} and {[}дз{]} as
voiced variants of х, ч, ц. Ukrainian has them as full-fledged
consonants---the ones that are an intrinsic part of a word and can
appear anywhere.
