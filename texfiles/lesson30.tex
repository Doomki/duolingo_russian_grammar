\chapter{Shopping}\label{shopping}

\section{Give me that!}\label{give-me-that}

By now, you have probably noticed a surprising lack of ``that one'' in
Russian. The thing is, Russian mostly uses ``этот'' both for ``this''
and ``that'', unless you need to make a contrast between ``this one
here'' and ``that one there''.

However, when you are really pointing at things, use whatever you like!

\begin{itemize}
\tightlist
\item
  «вот тут»\textasciitilde{}right here;«вот \'{э}тот» \textasciitilde{}
  ``this one here''
\item
  «вон там»\textasciitilde{}over there; «вон тот» \textasciitilde{}
  ``that one over there''
\end{itemize}

(``вот'' is acceptable with both)

\section{Clothing}\label{clothing}

\begin{itemize}
\tightlist
\item
  \textbf{од\'{е}жда} is a mass noun for ``clothes'', \textbf{\'{о}бувь} for
  footwear.
\item
  \textbf{т\'{у}фли} are also ``shoes'', but a more specific kind: ``dress
  shoes'' or the shoes you'd use with a gown
\item
  \textbf{бот\'{и}нок} \ldots{}a dictionary will say it's a bit higher than
  a ``dress shoe''. In reality, especially in men's speech, the word is
  often used for shoes, too
\item
  \textbf{сап\'{о}г} is most definitely a boot
\item
  \textbf{пальт\'{о}} is typically a long outer garment
\item
  \textbf{к\'{у}ртка} is more generic but usually refers to a shorter outer
  garment---with proportions not much different from a shirt
\item
  \textbf{руб\'{а}шка} is the word used for shirt. «Сор\'{о}чка» is a formal
  word for a shirt that is worn as a part of a suit (eg. with pants, a
  jacket and a necktie), but people still use ``руб\'{а}шка'' anyway.
\end{itemize}

\section{a bigger/smaller shirt}\label{a-biggersmaller-shirt}

From the \textbf{Adjectives} skill you might remember «б\'{о}льше» and
«меньше» as words for ``more/bigger'' and ``less/fewer/smaller''. Since
these work as adverbs, it is problematic to use them with nouns.

Instead, the words \textbf{«поб\'{о}льше» / «пом\'{е}ньше»} are used AFTER a
noun:

\begin{itemize}
\tightlist
\item
  Я хоч\'{у} стол поб\'{о}льше/пом\'{е}ньше.
\item
  Д\'{а}йте \'{я}блоко поб\'{о}льше.
\end{itemize}

This works with some other popular adjectives: подлинн\'{е}е, покор\'{о}че,
пол\'{у}чше. When not used with nouns directly, these have a colloquial
quality.

Actually, with adjectives other than \emph{больш\'{о}й/мал\'{е}нький} you may
resort to «б\'{о}лее дл\'{и}нное пальт\'{о}». However, «б\'{о}лее больш\'{о}е»? No. Just no.
