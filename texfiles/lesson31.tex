\chapter{Dative and Plurals}\label{dative-and-plurals}

\section{The Dative Case in
Russian}\label{the-dative-case-in-russian}

You have already seen that many \textbf{expressions of feelings and
experience} use the Dative: \emph{''Мне нр\'{а}вится\ldots{}'', ``М\'{а}ме
хорош\'{о}'', ``Ем\'{у} 5 лет'', ``Мне к\'{а}жется \ldots{}''} etc.

The Dative introduces an \textbf{\emph{indirect object}} of an action:
usually the person whom the action is directed towards. An example would
be a sentence like ``I handed a package to \emph{my mom}'': ``my mom''
here is a recipient.

Actually, this depends on the verb, just like in English. Some popular
verbs of speech, writing or ``giving'' will use the bare Dative for the
recipient: \textbf{\emph{говор\'{и}ть, сказ\'{а}ть, пис\'{а}ть, чит\'{а}ть, дать,
принест\'{и}}} and so on.

\section{Dative prepositions}\label{dative-prepositions}

\begin{itemize}
\tightlist
\item
  \textbf{по}: the basic meaning is ``movement along the
  surface''(``covering'' it) which may realise as ``walking around the
  park'', ``running down the street'', ``looking for it all over the
  house'' etc.
\item
  \textbf{к}: towards, to. Often used when you are going towards
  somebody (``towards Anna'' = «к Анне»)
\item
  several bookish prepositional phrases like «благодар\'{я}» (thanks to) or
  «вопрек\'{и}» (contrary to)
\end{itemize}

\textbf{По} has an additional meaning, ``apiece'' or ``each'' : «Он\'{и}
вз\'{я}ли по три \'{я}блока»=''They took 3 apples each''.

There is a bookish use of «по» meaning ``upon''. It goes with
Prepositional, and is mostly used in set prepositional phrases like «по
оконч\'{а}нии» (upon completion).

\section{Cases in plural}\label{cases-in-plural}

Plurals generally have only \textbf{one pattern} shared by all nouns.
The ending only depends on the case, not the class of a noun:, «я говор\'{ю}
о дом\'{а}х, стр\'{а}нах, город\'{а}х, \'{я}блоках, дочер\'{я}х».

Only the Nominative and (especially) the Genitive have a number of
different plural endings that depend on the class of a noun.
