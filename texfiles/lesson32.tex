\chapter{Speaking 1}\label{speaking-1}

\section{Sequence of tenses in
Russian}\label{sequence-of-tenses-in-russian}

There is no sequence of tenses in Russian whatsoever.

The information in a subordinate sentence is understood to be relative
to the main clause:

\begin{itemize}
\tightlist
\item
  Он сказ\'{а}л, что не зн\'{а}ет. = \emph{He said he didn't know.}
\end{itemize}

So if the piece of information is simply about where things are or what
someone does, use present tense in the subordinate clause.

\section{ли}\label{ux43bux438}

Use the particle ``ли'' in reported questions or situations when you
don't know which option is true:

\begin{itemize}
\tightlist
\item
  Я спрос\'{и}л, зн\'{а}ет ли он Москв\'{у}. = \emph{I asked him if he knew Moscow}.
\item
  Мы не зн\'{а}ем, б\'{у}дет ли он в \'{о}фисе. = \emph{We don't know whether he is
  going to show up in the office}.
\end{itemize}

The particle is attached to the word that is in doubt. It needn't be a
verb, for instance, «Я не зн\'{а}ю, в Москв\'{е} ли он» (i.e. whether he is in
Moscow or in some other city). The particle generally attaches to the
first stressed word of the clause.

\section{Talk or say?}\label{talk-or-say}

The verb \emph{говор\'{и}ть} is used both as ``to say, to tell'' and as ``to
talk, to speak''. When you report someone's words, obviously, the 2nd
meaning is in action:

\begin{itemize}
\tightlist
\item
  Она говор\'{и}т, что х\'{о}чет спать. = She says that she wants to sleep.
\end{itemize}

Russian has a whole set of \textbf{perfective} verbs. The thing is,
usually you arrange verbs neatly into closely matching pairs of
\emph{imperfective + perfective}. And these are different for the two
meanings of «говор\'{и}ть»:

\begin{itemize}
\tightlist
\item
  to say $ \rightarrow$ говор\'{и}ть / ска\'{з}ать
\item
  to speak $ \rightarrow$ говор\'{и}ть / поговор\'{и}ть
\end{itemize}

Remember «Скаж\'{и}те, пож\'{а}луйста \ldots{} » ?

Rather than referring to ongoing actions or past(future) actions in
general, perfective verbs refer to actions in a point-wise manner,
ignoring the action's inner structure. That is, such ``singular''
actions happen at some particular ``moment'' and can be conveniently
arranged in a sequence when telling a story. This distinction is about
to come into focus in one of the following skills.
