\chapter{Perfective Verbs -1}\label{perfective-verbs--1}

\section{Aspect in Russian}\label{aspect-in-russian}

Verbs in Russian come in two `flavors' : \emph{perfective} (eg.
``пригот\'{о}вить'') and \emph{imperfective} (eg. ``гот\'{о}вить'').

Let's get this straight right away: most perfectives are made by
attaching a prefix---and the endings of the resulting verb change in the
same way they changed for the source verb.

\textbf{Perfective} verbs express an action, an ``event'' linked to a
point in time. Sometimes they assert the presence of a result. You use
them for sequences of actions, too.

\textbf{Imperfective} verbs are used for everything else: processes,
states, repeated actions and for generic reference to an action (when
the time of occurence is irrelevant).

In this introductory lesson we deal with the most basic pattern of use:

\begin{itemize}
\tightlist
\item
  \textbf{perfective verbs} are used to tell stories about
  \textbf{successive} actions
\item
  imperfectives are used for \emph{simultaneous} processes
\item
  \textbf{perfective verbs} are often used to describe \emph{single}
  actions that have a specific result, e.g., ``Give me that'', ``I
  bought some food'', ``I have painted many pictures''. However, not all
  of them can be reliably described like that.
\item
  we use imperfective to tell that someone has or has never done
  something, especially in ``out of the blue'' situations. When the
  action was \emph{supposed} to be done (which is known by listener), we
  use the perfective.
\end{itemize}

\section{Morphology}\label{morphology}

Being too lazy to make up many different verbs, we usually make new ones
based on the old ones. The vast majority of unprefixed verbs are
imperfective.

\begin{itemize}
\tightlist
\item
  \textbf{Prefixation} is the main method to create a perfective verb:
  пис\'{а}ть$ \rightarrow$напис\'{а}ть, идт\'{и}$ \rightarrow$пойт\'{и}.
\item
  a different \textbf{suffix} is sometimes used: оп\'{а}здывать$ \leftarrow$опозд\'{а}ть
\item
  occasionally, the \textbf{stress} changes: нарез\'{а}ть$ \rightarrow$нар\'{е}зать
\item
  \textbf{different stems} are used for a few verbs: говор\'{и}ть$ \rightarrow$сказ\'{а}ть
\end{itemize}

The last phenomenon is know as \emph{suppletion} and only happens for a
limited number of verbs and their derivatives. The English verb ``to
go'' is another example of such behavior (its past for is ``went'').

Note that suffixation is very popular for \emph{secondary}
imperfectives. Usually only one prefixed verb is considered an ``ideal
match'' for an imperfective verb. Others are somewhat different in
meaning (or a lot different). But you need imperfective partners for
these, too, so Russian uses suffixes for that:

\begin{itemize}
\tightlist
\item
  чит\'{а}ть = to read (imperf.)
\item
  перечит\'{а}ть = to reread (perf.) $ \rightarrow$ \emph{cannot be considered a
  ``natural'' perfective for this verb}
\item
  переч\'{и}тывать = to reread (imperf.)
\end{itemize}

\section{can}\label{can}

The verb «мочь» is used to talk about the general possibility of
something, and also, very often---about your ability to perform
something and reach some result. Perfectives are used in the second
meaning:

\begin{itemize}
\tightlist
\item
  Я мог\'{у} напис\'{а}ть кн\'{и}гу за м\'{е}сяц = I can write a book in a month.
\item
  Он\'{а} м\'{о}жет посмотр\'{е}ть? = Can she take a look?
\end{itemize}

We do not use мочь for skills. Russian has \textbf{уметь} for this.

\section{опять /
снова}\label{ux43eux43fux44fux442ux44c-ux441ux43dux43eux432ux430}

Both mean ``again'' and are largely interchangeable when they mean that
an action from the past occurs again.

«Опять» is more popular but it's focused on staying ``the same as
before''. «Снова» (cf. «новый») can also mean action performed ``anew,
from the beginning''.

Only «опять» is used in «опять же» (\textasciitilde{}``besides'').

When asking someone to repeat, use «ещё раз».

\section{What else is there to
it?}\label{what-else-is-there-to-it}

\textbf{imperfective verbs}

\begin{itemize}
\tightlist
\item
  name the action as a whole (``I can \emph{swim}'')
\item
  describe prolonged \textbf{\emph{states}} and
  \textbf{\emph{processes}}, regular actions
\end{itemize}

\textbf{Perfective verbs} describe \textbf{events}: singular, definite
actions that are viewed as localized in time. They ``happened'' at some
moment (``I made a video'', ``I slept for some time and then went
outside''). Or they describe a certain change of state at some ``turning
point'' (not yet eaten$ \rightarrow$eaten, not slept enough$ \rightarrow$slept enough and ready to
get up).

It is argued in a few works that ``a natural'' perfective is just a
prefixed verb where a prefix's metaphorical meaning so conveniently
overlaps the verb's own meaning, that you cannot feel any change. So
don't be surprised if some vague actions have several perfective matches
for a single imperfective verb.

That also means that sometimes you'd better memorize a pair even if it
is technically a ``poor'' match. After all, in some contexts it will
come in handy:

\begin{itemize}
\tightlist
\item
  есть $ \rightarrow$ съ\'{е}сть (to consume something, completely)
\item
  есть $ \rightarrow$ по\'{е}сть (to have a meal, to spend some time eating---regardless
  of whether you finish your meal or decide you've had enough half-way)
\end{itemize}
