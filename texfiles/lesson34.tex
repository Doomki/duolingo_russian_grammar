\chapter{Instrumental Case}\label{instrumental-case}

\section{Fortunately, this is the very last
case!}\label{fortunately-this-is-the-very-last-case}

It is used for some very specific meanings, that's why we've put off
covering it for so long.

\begin{itemize}
\tightlist
\item
  It is used alone for a ``tool'' or an ``agent'' of an action. English
  mostly uses ``with'' or ``by'' instead: ``молотк\textbf{ом}'' (with
  hammer), ''ветр\textbf{ом}'' (by wind), ``сил\textbf{ой}'' (by force)
\item
  It is used alone with some verbs of ``being'', ``becoming'',
  ``seeming'': \emph{Я стал учител\textbf{ем}} \textasciitilde{} ``I
  have become a teacher''
\end{itemize}

It is also used with prepositions: - \textbf{с (со)} = ``with''
(together with someone/something) --- note that with prepositions ``Я с
ней'' or even ``Мы с ней'' is the most natural way of saying ``She and
I'' - \textbf{за}/ \textbf{перед} --- behind/ in front of -
\textbf{над}/ \textbf{под} --- above/under - \textbf{между} --- between
(also used with Genitive)

\section{Мы с
тобой}\label{ux43cux44b-ux441-ux442ux43eux431ux43eux439}

When you tell someone about ``you and I'' or ``my friend and I'' etc.,
it is most idiomatic to use \textbf{мы с} + your companion in
Instrumental.

\begin{itemize}
\tightlist
\item
  Мы с тобой друзья. = You and I are friends.
\item
  Мы с мамой вчера купили компьютер. = Mom and I bought a computer
  yesterday.
\end{itemize}

Of course, when translating sentences out of the blue, you cannot
(strictly speaking) tell if a speaker means ``I'' or ``we''. This is
rarely a source of confusion in real situations (where it is unlikely a
speaker goes on randomly switching between ``I'' and ``we'' ).

Sometimes you can interpret a joint action using ``and'' or ``with'',
whatever sounds more natural:

\begin{itemize}
\tightlist
\item
  Мы с ними вчера не говорили. = They and I didn't talk yesterday / I
  didn't talk with (to) them yesterday.
\end{itemize}

\section{But wait, there's more!}\label{but-wait-theres-more}

Actually, Russian also has a handful of inconsistent cases that only
exist for some words. They are (mostly) beyond the scope of this course:

\begin{itemize}
\tightlist
\item
  \textbf{\emph{the Locative-2: the most important}} Why? Because it's
  obligatory with the nouns that it applies to. It expresses the meaning
  of place, with \emph{в},\emph{на} or both. It exists for over a
  hundred masculine nouns: в год\textbf{у}, на мост\textbf{у}, в
  лес\textbf{у}, на пол\textbf{у}. And for about 20 feminine nouns in
  ---\textbf{ь}: в кров\textbf{и} (the ending is always stressed for
  both!)
\item
  \textbf{\emph{the Neo-vocative:}} a form of a name used when
  addressing a person. It exists for common names and several nouns:
  \textbf{\emph{Вань! Вер! Алён! Мам! Пап!}} (just the last vowel sound
  is removed). The Historical vocative (``человек $ \rightarrow$ человече'') has been
  lost in modern Russian.
\item
  \textbf{\emph{the Genitive-2}} for ``some amount of substance''.
  Increasingly replaced by the usual Genitive but still can be used for
  several masculine nouns: ``Хочу ча\textbf{ю}''
\item
  \textbf{\emph{``Waiting'' case:}} not much of a case, but actually the
  verb ``ждать'' (to wait) would use Accusative for people and things
  that can affect their appearance and Genitive for everything else (an
  event/thing that does not choose when to arrive).
\end{itemize}
