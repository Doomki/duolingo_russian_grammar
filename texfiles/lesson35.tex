\chapter{Body, Life and Death}\label{body-life-and-death}

\section{You have big eyes}\label{you-have-big-eyes}

Be careful NOT to use «есть» when describing properties of body parts,
if their existence is normal and unlikely to surprise anyone:

\begin{itemize}
\tightlist
\item
  У меня длинный нос = \emph{I have a long nose}.
\end{itemize}

\section{A Handy thing to know}\label{a-handy-thing-to-know}

The Russian words for limbs and what they have on the end of them can be
a little confusing initially. Depending on the situation, рука can mean
hand or it can mean arm. The same is true for нога; it can be foot or
leg. Most of the time the meaning is clear from the context.

The difference works as follows:

\begin{itemize}
\tightlist
\item
  in English ``hand'', strictly speaking, means the part starting at the
  wrist.
\item
  in Russian «рука», strictly speaking, means the whole arm.
\item
  when you need a word describing the limb \emph{functionally} (as
  something we use to work, to grab things etc.), English uses ``hand''
  while Russian uses ``arm''
\end{itemize}

The same goes for the lower limb.
