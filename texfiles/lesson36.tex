\chapter{Genitive Plural}\label{genitive-plural}

\section{The formation of Genitive
Plural}\label{the-formation-of-genitive-plural}

All other forms (except the Nominative) are the same for all plural
nouns, regardless of gender. The Genitive is the other exception. Here
is how it is formed:

\begin{itemize}
\item
  \textbf{-а, -я, -о nouns}: just remove the last vowel sound. Extends
  to \textbf{-ия} and \textbf{-ие} nouns (which become \textbf{-ий}). A
  vowel is inserted if a consonant cluster forms at the end. We will
  address a few common cases of \emph{fleeting vowels} later in the
  course.

  \begin{itemize}
  \tightlist
  \item
    м\'{а}ма, нед\'{е}ля $ \rightarrow$ \textbf{мам}, нед\'{е}ль
  \item
    сл\'{о}во, окн\'{о} $ \rightarrow$ \textbf{слов}, \'{о}кон
  \item
    фам\'{и}лия, мел\'{о}дия $ \rightarrow$ фам\'{и}лий, мел\'{о}дий
  \item
    к\'{о}шка $ \rightarrow$ к\'{о}шек
  \end{itemize}
\item
  \textbf{hard consonant}: typical ``masculine'' nouns ending in
  \emph{hard} non-sibilant consonants get the ending \textbf{-ов}. Those
  in ``-й'' get \textbf{-ев}, and so do nouns in ``-ц'' when the ending
  is unstressed (which won't help you much).

  \begin{itemize}
  \tightlist
  \item
    стол, от\'{е}ц, г\'{о}род $ \rightarrow$ \textbf{стол\'{о}в}, отц\'{о}в, город\'{о}в
  \item
    м\'{е}сяц $ \rightarrow$ м\'{е}сяцев
  \end{itemize}
\item
  \textbf{soft consonant}: feminine and masculine nouns ending in
  \textbf{-ь} or hushes (Ж, Ш, Щ, Ч) will get \textbf{-ей} as the
  ending. Neuter nouns ending in \textbf{-е} also use this pattern.

  \begin{itemize}
  \tightlist
  \item
    ночь, кров\'{а}ть $ \rightarrow$ \textbf{ноч\'{е}й}, кров\'{а}тей
  \item
    уч\'{и}тель, муж $ \rightarrow$ учител\'{е}й, муж\'{е}й
  \item
    м\'{о}ре $ \rightarrow$ \textbf{мор\'{е}й}
  \end{itemize}
\item
  «\'{и}мя» and «вр\'{е}мя» become \textbf{имён, времён} (though, for «с\'{е}мя» and
  «стр\'{е}мя» it is «сем\'{я}н» and «стрем\'{я}н»)
\end{itemize}

\section{Not so easy}\label{not-so-easy}

\begin{itemize}
\item
  be careful around nouns that form plurals irregularly, like друг
  $ \rightarrow$\textbf{друзь\'{я}}. Here are the genitive plurals of «друг», «мать»,
  «дочь», «сын», «стул», «брат», «лист» and «д\'{е}рево» :

  \begin{itemize}
  \tightlist
  \item
    \textbf{друз\'{е}й, матер\'{е}й, дочер\'{е}й, сынов\'{е}й}
  \item
    \textbf{ст\'{у}льев, бр\'{а}тьев, л\'{и}стьев, дер\'{е}вьев}
  \end{itemize}
\end{itemize}
