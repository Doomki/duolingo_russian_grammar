\chapter{\texorpdfstring{Name and polite
``You''}{Name and polite You}}\label{name-and-polite-you}

\section{Thou art}\label{thou-art}

Russian makes a distinction between \textbf{ты}, singular ``you'', and
\textbf{вы}, plural ``you'' (y'all). The latter also doubles for
``polite'' you, with verbs also in plural. And don't forget that the
``excuse'' in ``Excuse me'' is a verb!

\begin{itemize}
\tightlist
\item
  Use \textbf{ты} with friends and your family members
\item
  Use \textbf{вы} with adult strangers, your teachers and in other
  formal interactions (at the store, the doctor's, the airport etc.)
\item
  People use \textbf{вы} with those who are much older
\item
  Nobody is ``polite'' toward kids
\end{itemize}

Contrary to what many native speakers have come to believe in the last
ten or fifteen years, the polite ``you'' is not automatically
capitalized in Russian, and never was. Such capitalization is used in
some formal styles.

\section{Grandson, son of Grand}\label{grandson-son-of-grand}

As you might know if you ever read any Russian literature, Russians have
three names; their first name and their surname---just like you
have---and a \emph{patronymic} (отчество), which is based on their
father's name (отец = father). A very common `polite' pattern is to use
a person's first name and a patronymic:

\begin{itemize}
\tightlist
\item
  Ив\'{а}н Иванович, вы з\'{а}няты? = \emph{Ivan Ivanovich, are you busy?}
\end{itemize}

In this course, \textbf{name+patronymic} are always used with the polite
\textbf{вы}-form.

\section{What is your name?}\label{what-is-your-name}

«Как вас зов\'{у}т?» is literally ``How (do) they call you?''

Russian has a casual diminutive form for many common names, : Ив\'{а}н$ \rightarrow$В\'{а}ня,
Мар\'{и}я$ \rightarrow$Маша, Алекс\'{а}ндр(Алекс\'{а}ндра)$ \rightarrow$С\'{а}ша, Евг\'{е}ний(Евг\'{е}ния)$ \rightarrow$Ж\'{е}ня,
Ел\'{е}на$ \rightarrow$Л\'{е}на, Алекс\'{е}й$ \rightarrow$Лёша, Пётр$ \rightarrow$П\'{е}тя. Needless to say, there's no
``politeness'' with these, but they are often used with some degree of
affection.

\section{Excuse me\ldots{}}\label{excuse-me}

Russian has two very common polite patterns for questions that English
does not:

\begin{itemize}
\tightlist
\item
  \textbf{negative} questions give a shade of ``by any chance'':
  «Извин\'{и}те, вы не зн\'{а}ете Миха\'{и}ла?» = \emph{Excuse me, do you happen to
  know Mikhail?}
\item
  \textbf{``Please tell''} when asking for information: «Скаж\'{и}те,
  пож\'{а}луйста, где муз\'{е}й?» = \emph{Excuse me, where is the museum?}
\end{itemize}

\section{Thank you}\label{thank-you}

«Спас\'{и}бо» is the word. A fancier option would be «Благодар\'{ю}!» (a form of
the verb «благодар\'{и}ть», ``to thank''), though quite a number of people
use it, if only for variety.
