\chapter{Plurals}\label{plurals}

Here is how the Nominative Plural is formed.

\begin{longtable}[]{@{}lll@{}}
\toprule
TYPE & ending & Example\tabularnewline
\midrule
\endhead
\textbf{-а/-я} -nouns & \textbf{ы/и} & м\'{а}мы, з\'{е}мли\tabularnewline
\textbf{-ь} -nouns, feminine & \textbf{и} & кров\'{а}ти\tabularnewline
most \textbf{consonant-ending} masculines & \textbf{ы/и} & стол\'{ы},
м\'{а}льчики\tabularnewline
\textbf{-о/-е} -nouns & \textbf{а/я} & \'{о}кна, мор\'{я}\tabularnewline
some \textbf{consonant-ending} masculines & \textbf{а/я} & доктор\'{а},
глаз\'{а}\tabularnewline
\bottomrule
\end{longtable}

(so, the plural «\'{я}блоки» is actually an uncommon way of doing it)

There are some irregular plurals too.

\section{Spelling Rules}\label{spelling-rules}

Or maybe not. Sometimes Russian forces your choice of vowel to spell or
pronounce after a certain letter.

\textbf{The 7-letter rule}: Whenever you make any form of a word, and
you need to write И or Ы, check this:

\begin{itemize}
\tightlist
\item
  after К, Г, Х and Ш, Ж, Щ, Ч always use \textbf{И}
\end{itemize}

These are velars (``back'' consonants) and hushes. For hushes, it is
merely a spelling convention, owing to their former ``soft'' status. For
velars, it is true to their pronunciation --- i.e., these consonants
always use the palatalized И where another consonant would use Ы:

\begin{itemize}
\tightlist
\item
  стран\'{а} $ \rightarrow$ стр\'{а}ны
\item
  строк\'{а} $ \rightarrow$ стр\'{о}ки
\end{itemize}

Of these seven consonants, «К» should be your main concern for now. A
lot of nouns have it as a suffix or a part of their suffix, forcing you
to remember this rule.

\textbf{The 8-letter rule}: Whenever you make any form of a word, and
you need to write А, У or Я, Ю after a consonant, follow the rule:

\begin{itemize}
\tightlist
\item
  after К, Г, Х , Ш, Ж, Щ, Ч \textbf{and Ц}, always use \textbf{А} or
  \textbf{У}
\end{itemize}
