\chapter{Where is it?}\label{where-is-it}

Russian words take different forms depending on their role in the
sentence. These forms are called \textbf{cases}. A few forms may look
the same (cf. ``frequent rains'' vs. ``It rains often'').

These forms have names (mostly calques from Latin) that describe some
``prototypal'' use of such case: Nominative, Accusative, Genitive,
Prepositional, Dative and Instrumental. For you, these are just tags:
the use is what defines a case.

As of now, you know the NOMINATIVE case: the dictionary form of a word.
This form acts as the \emph{grammatical subject} of the sentence, the
``doer''. It is also used for both nouns in ``A is B'' structure:

\begin{itemize}
\tightlist
\item
  Мой п\'{а}па ест.
\item
  Том --- мой брат.
\end{itemize}

You also know a few Genitive forms (у \textbf{меня}) but that's it. For
now, we will tackle something easier.

\section{Prepositional case}\label{prepositional-case}

When we talk about things \textbf{\emph{being}} somewhere, we typically
use \textbf{в}(in) or \textbf{на} (on) with the Prepositional form of
the noun. It doesn't work when you mean \emph{motion} to that place!

The Prepositional case (a.k.a. Locative) is the only case that is
\emph{never} used on its own without a preposition, even though only
four or five prepositions ever use it:

\begin{itemize}
\tightlist
\item
  Я на конц\'{е}рте. = I am at a concert.
\item
  Я в шк\'{о}ле. = I am at (in) school.
\item
  в\'{и}део о шк\'{о}ле = a video about school
\end{itemize}

Unlike English (``at/in school''), in Russian each ``place'' is
associated with just one preposition. The rough overall rule is: use
``в''(in, at) when talking about buildings and places with certain
boundaries and use ``на'' (on, at) when talking about open spaces or
events:

\begin{itemize}
\tightlist
\item
  в д\'{о}ме \emph{(at home)}, в шк\'{о}ле \emph{(at school)}, в к\'{о}мнате
  \emph{(in the room)}, в те\'{а}тре \emph{(in the theater)}, в кин\'{о}
  \emph{(at the cinema)}, в университ\'{е}те \emph{(at the university)}
\item
  на ул\'{и}це \emph{(in the street, outdoors)}, на пл\'{о}щади \emph{(at the
  square)}, на конц\'{е}рте \emph{(at the concert)}, на ур\'{о}ке \emph{(at the
  lesson)}, на кор\'{а}бле \emph{(on a ship)}
\end{itemize}

When you mean physically being inside/on top of some object, there is
little ambiguity. ``Places'', unfortunately, require memorization.

\section{Prepositional endings}\label{prepositional-endings}

Here is the rule that covers most nouns:

\begin{itemize}
\tightlist
\item
  feminine nouns ending in ь take \textbf{-и}
\item
  nouns ending in -ия, -ий or -ие also take \textbf{-и} (so that they
  end in -ии instead)
\item
  all other nouns take \textbf{-е}
\end{itemize}

\section{What about me and my
friends?}\label{what-about-me-and-my-friends}

Use ``у + Genitive'' when talking about being at some \emph{person's}
place: Да, я у др\'{у}га = \emph{Yeah, I am at my friend's place}.

\section{$ \blacktriangle$$ \blacktriangledown$WC}\label{wc}

The room with a toilet is \textbf{туал\'{е}т}. In this course, we stick to
the North American ``bathroom'', even though a room with a bath is,
technically в\'{а}нная (it has в\'{а}нна, ``a bath''). Still, in Russian you
would not ask for a ``bath-room'' unless you really mean it.

\section{\texorpdfstring{And what if I gotta go
\textbf{away}?}{And what if I gotta go away?}}\label{and-what-if-i-gotta-go-away}

We'll deal with that later. But the pattern is consistent. When you
\textbf{\emph{are}} somewhere, going \textbf{\emph{to}} that place and
going \textbf{\emph{away}} from that place, use the following triplets:

\begin{longtable}[]{@{}lll@{}}
\toprule
AT & TO & FROM\tabularnewline
\midrule
\endhead
\textbf{в} + Prep & \textbf{в} + Acc & \textbf{из} + Gen\tabularnewline
\textbf{на} + Prep & \textbf{на} + Acc & \textbf{с} + Gen\tabularnewline
\textbf{у} + Gen & \textbf{к} + Dat & \textbf{от} + Gen\tabularnewline
\bottomrule
\end{longtable}

For example, if the place is used with на, the correct prepositions for
the three uses are на--на--с.
