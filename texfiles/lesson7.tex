\chapter{Animals 1}\label{animals-1}

\section{\texorpdfstring{``Spelling
rules''}{Spelling rules}}\label{spelling-rules}

Note how plurals of «соб\'{а}ка» and «к\'{о}шка» end in И: соб\'{а}ки, к\'{о}шки, even
though you might expect А to turn into Ы.

There are some restrictions on which consonants are used with which
vowels when making word forms. Here are the rules for \textbf{и, а, у}
vs. \textbf{ы, я, ю}:

\begin{itemize}
\tightlist
\item
  use only И, not Ы, after \textbf{к, г, х}/ \textbf{ж, ш, щ, ч}
\item
  use only А, У after \textbf{к, г, х}/ \textbf{ж, ш, щ, ч} and
  \textbf{ц} (and never use Я, Ю after them)
\end{itemize}

\emph{К, Г, Х} are called velar consonants (i.e. made in the back) and
\emph{Ш, Щ, Ж, Ч} are often called \emph{hushes}. The latter do not show
palatalized/non-palatalized pairs in modern Russian, so the spelling
rule does not affect pronunciation anyhow. It's just a convention.

\section{Fleeting vowels}\label{fleeting-vowels}

It is not too important for you at the moment, but you may notice how О
and Е sometimes appear in consonant clusters or disappear from them. For
example:

\begin{itemize}
\tightlist
\item
  \'{Э}то лев. = This is a lion.
\item
  В зооп\'{а}рке нет льв\'{а}. = There isn't a lion at the zoo.
\end{itemize}

\emph{Later you will encounter the Genitive plural (often used with
numbers and words like ``many'' or ``few''), which shows a simple
pattern for \textbf{-к}-suffixed feminine nouns that do not have a vowel
before ``-ка'':}

\begin{itemize}
\tightlist
\item
  много кошек = \emph{many cats}
\item
  много девочек = \emph{many girls}
\item
  много уток = \emph{many ducks}
\item
  много тарелок = \emph{many plates}
\end{itemize}

\emph{As you can see, the vowel (О or Е) depends on whether the previous
consonant is palatalized or not. Hushes behave as if they were
palatalzed, despite Ж and Ш having lost this quality in the modern
language.}
