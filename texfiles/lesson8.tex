\chapter{Genitive Case - 1}\label{genitive-case---1}

In Russian ``I have'' is expressed by «У меня (есть)» structure. The
owner is in the \emph{Genitive case}.

\section{Genitive}\label{genitive}

"The \textbf{of}-case". It is one of the most universal cases. How do
you make the forms? Here is the regular pattern:

\begin{longtable}[]{@{}llllll@{}}
\toprule
\begin{minipage}[b]{0.16\columnwidth}\raggedright\strut
\textbf{\emph{ENDING}}\strut
\end{minipage} & \begin{minipage}[b]{0.16\columnwidth}\raggedright\strut
\strut
\end{minipage} & \begin{minipage}[b]{0.16\columnwidth}\raggedright\strut
\textbf{Genitive sg.}\strut
\end{minipage} & \begin{minipage}[b]{0.16\columnwidth}\raggedright\strut
\strut
\end{minipage} & \begin{minipage}[b]{0.16\columnwidth}\raggedright\strut
soft stem\strut
\end{minipage} & \begin{minipage}[b]{0.16\columnwidth}\raggedright\strut
\strut
\end{minipage}\tabularnewline
\midrule
\endhead
\begin{minipage}[t]{0.16\columnwidth}\raggedright\strut
\textbf{-a/-я}\strut
\end{minipage} & \begin{minipage}[t]{0.16\columnwidth}\raggedright\strut
мама\strut
\end{minipage} & \begin{minipage}[t]{0.16\columnwidth}\raggedright\strut
мам\textbf{ы}\strut
\end{minipage} & \begin{minipage}[t]{0.16\columnwidth}\raggedright\strut
\strut
\end{minipage} & \begin{minipage}[t]{0.16\columnwidth}\raggedright\strut
земля\strut
\end{minipage} & \begin{minipage}[t]{0.16\columnwidth}\raggedright\strut
земл\textbf{и}\strut
\end{minipage}\tabularnewline
\begin{minipage}[t]{0.16\columnwidth}\raggedright\strut
\textbf{zero-ending} \emph{masc}, \textbf{-о/-е} \emph{neut}\strut
\end{minipage} & \begin{minipage}[t]{0.16\columnwidth}\raggedright\strut
сок / молоко\strut
\end{minipage} & \begin{minipage}[t]{0.16\columnwidth}\raggedright\strut
сок\textbf{а} / молок\textbf{а}\strut
\end{minipage} & \begin{minipage}[t]{0.16\columnwidth}\raggedright\strut
\strut
\end{minipage} & \begin{minipage}[t]{0.16\columnwidth}\raggedright\strut
конь\strut
\end{minipage} & \begin{minipage}[t]{0.16\columnwidth}\raggedright\strut
кон\textbf{я}\strut
\end{minipage}\tabularnewline
\begin{minipage}[t]{0.16\columnwidth}\raggedright\strut
\textbf{-ь} \emph{fem}\strut
\end{minipage} & \begin{minipage}[t]{0.16\columnwidth}\raggedright\strut
мышь\strut
\end{minipage} & \begin{minipage}[t]{0.16\columnwidth}\raggedright\strut
мыш\textbf{и}\strut
\end{minipage} & \begin{minipage}[t]{0.16\columnwidth}\raggedright\strut
\strut
\end{minipage} & \begin{minipage}[t]{0.16\columnwidth}\raggedright\strut
\strut
\end{minipage} & \begin{minipage}[t]{0.16\columnwidth}\raggedright\strut
\strut
\end{minipage}\tabularnewline
\bottomrule
\end{longtable}

\emph{A \textbf{zero} ending means that the word ends in a consonant or
a soft sign (which is just a way to show the final consonant is
``soft''). In the Nominative singular, a Russian word can only have the
following endings:} \textbf{а, я, о, е, ё} \emph{or \textbf{nothing}
(``zero ending'').}

\section{Genitive of Negation}\label{genitive-of-negation}

If you use «нет» to say that there is ``no'' something or you do not
have it, the object is always in Genitive:

У мен\'{я} есть \'{я}блоко $ \rightarrow$ У мен\'{я} нет \textbf{\'{я}блока}

Здесь есть рюкз\'{а}к $ \rightarrow$ Здесь нет \textbf{рюкзак\'{а}}.

\section{Major uses}\label{major-uses}

\begin{itemize}
\tightlist
\item
  \textbf{``of'' (possession)}: яблоко мам\textbf{ы} = mom's apple
\item
  \textbf{``of'' (amount)}: чашка ча\textbf{я}, много ча\textbf{я} = a
  cup of tea, a lot of tea
\end{itemize}

A huge number of prepositions requires this case. Yes, «у меня есть», «У
неё есть» only use «меня» and «неё» because «у» wants Genitive.

For \emph{он}, \emph{она} and \emph{оно} Genitive doubles as a
non-changing possessive ``his'', ``her'', ``their'': \textbf{его, её,
их}.

\begin{itemize}
\tightlist
\item
  initial «н» is used for him/her/them with the majority of prepositions
  (doesn't affect possessives)\\
\end{itemize}

\section{Indeclinable nouns}\label{indeclinable-nouns}

A little side note: some nouns of foreign origin are indeclinable. It
means that all their forms are the same. Foreign nouns that end in о/е
become like that (кофе, метро, радио, резюме), as well as all nouns that
do not fit into Russian declension patterns (see above).

This includes female names that end in anything other than А or Я. A few
\textbf{-ь}-ending names are an exception (Любовь and Biblical names
like Юдифь).

So, all of the following names are automatically indeclinable: Маргарет,
Мэри, Элли, Дженни, Рэйчел, Натали, Энн, Ким, Тесс, Жасмин.

\section{I am away}\label{i-am-away}

Russian also uses the Genitive to state that someone is ``away'', ``not
there'': \emph{Мамы сейчас нет}. In English such use would correspond to
``There is no mom at the moment'', or even ``There is no me now''. We
are not hard on that particular construction in the course, but it is
important to know it all the same.

\emph{Added bonus: when a verb directly acts on a noun, the noun is
called} \textbf{\emph{a direct object}} \emph{and is in Accusative. In
Russian, only -а/-я feminine nouns have a unique form for it. Others
just reuse Genitive or don't change the word at all (Nominative)}

\section{Nothing}\label{nothing}

Russian uses\ldots{}. let's call it ``consistent'' negation. It means
that in negative sentences you are \emph{required} to use ``nothing''
instead of ``anything'', ``nowhere'' instead of ``somewhere'' and so on.
Let's meet the first of these pronouns:

\begin{itemize}
\tightlist
\item
  У меня \textbf{ничего} нет. = I don't have anything.
\item
  Она \textbf{ничего} не ест. = She doesn't eat anything.
\end{itemize}

You'll also notice that, unlike standard English, Russian has no rule
against using double negatives.
