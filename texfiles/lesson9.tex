\chapter{Possessives and Gender}\label{possessives-and-gender}

\section{Russian possessives}\label{russian-possessives}

There isn't much to say about words like ``my'' or ``your'' in Russian.

\begin{itemize}
\item
  \textbf{his/her/their} do not change: \textbf{ег\'{о}, её, их} (and they
  don't get an initial Н after prepositions!)
\item
  \textbf{my/your/our} roughly follow an adjectival pattern, i.e. they
  copy the gender and the case of the noun they describe. Just like
  \textbf{этот}:

  \begin{quote}
  \begin{itemize}
  \tightlist
  \item
    мой/твой/наш папа
  \item
    мо\textbf{\'{я}}/тво\textbf{\'{я}}/н\'{а}ш\textbf{а} м\'{а}ма
  \end{itemize}
  \end{quote}
\end{itemize}

Unlike English, no distinction is made between \emph{my} and
\emph{mine}, \emph{her} and \emph{hers} etc.

\textbf{Pronunciation}: in «его», as well as in adjective endings and
``сегодня'' the letter \textbf{Г} is pronounced \textbf{В}. It is a
historical spelling.

\section{Grammatical gender}\label{grammatical-gender}

Nouns in Russian belong to one of three genders: feminine, masculine or
neuter. If a noun means a person of a certain gender, use that one. For
all other nouns look at the end of the word:

\begin{longtable}[]{@{}lll@{}}
\toprule
ending in Nom.sg. & gender & examples\tabularnewline
\midrule
\endhead
\textbf{-а/-я} & feminine & м\'{а}ма, земл\'{я}, Росс\'{и}я, маш\'{и}на\tabularnewline
\textbf{consonant} & masculine & сок, м\'{а}льчик, чай, интерн\'{е}т,
апельс\'{и}н\tabularnewline
\textbf{-о/-е} & neuter & окн\'{о}, яйц\'{о}, м\'{о}ре\tabularnewline
\textbf{-ь} & feminine or masculine; consult a dictionary & л\'{о}шадь,
ночь, мать, люб\'{о}вь / день, конь, медв\'{е}дь, уч\'{и}тель\tabularnewline
\bottomrule
\end{longtable}

If there's a soft sign, it isn't possible to predict the gender, at
least, not accurately. However, about 65-70\% of the most used nouns
that end in \textbf{-ь} are feminine. Also, you can learn the common
suffixes ending in a soft sign that produce a word of a predictable
gender. They are:

\begin{itemize}
\tightlist
\item
  \textbf{-ость/-есть\emph{,} -знь} $ \rightarrow$ feminine
\item
  \textbf{-тель, -арь, -ырь} $ \rightarrow$ masculine
\end{itemize}

\begin{center}\rule{0.5\linewidth}{\linethickness}\end{center}

\emph{All nouns with \textbf{-чь, щь, -шь, -жь} at the end are feminine.
The convention is to spell feminine nouns with a soft sign and masculine
ones without one: нож, луч, муж, душ. It doesn't affect pronunciation,
anyway.}
